Nowadays seismic hazard analysis serves different needs coming 
from a wide spectrum of users and applications. 
%
These may encompass engineering design, assessment of earthquake risk 
to portfolios of assets within the insurance and reinsurance sectors, 
engineering seismological research, and effective mitigation via public 
policy in the form of urban zoning and building design code formulation.

Decisions made using the results of seismic hazard assessment involve
citizens and capitals and may have large impacts on our day-to-day 
life. 

Seismic hazard is still an active research area in science and engineering.

The development process of software for seismic hazard assessment 
must be able to provide the flexibility needed for the progress of 
science and to guaratee that the results generated comply with basic
quality standards.

Quality Control is a set of activities designed to evaluate whether a 
developed product (project document, developed system etc.) meets 
customer requirements. It ensures that delivered products are checked 
for quality and determines how well it is built. 
Its focus is to find defects and to ensure that they are corrected.
(from
http://www.planit.net.au/resource/software-quality-assurance-is-it-the-same-as-testing/)

\section{Quality Assurance}


From the IEEE ``Standard for Software Quality Assurance Processes'':

\emph{Software quality assurance is a set of activities that define and 
assess the adequacy of software processes to provide evidence that establishes 
confidence that the software processes are appropriate for and produce 
software products of suitable quality for their intended purposes. 
A key attribute of SQA is the objectivity of the SQA function with 
respect to the project. The SQA function may also be organizationally 
independent of the project; that is, free from technical, managerial, 
and financial pressures from the project.}
