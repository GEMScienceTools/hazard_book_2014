In this section we illustrate the benchmark tests implemented in the 
\gls{acr:oqe}. These tests are organised in two main sections. In the 
first one we consider simple testing cases based on a single seismic 
source, in the second set we will describe tests designed to check 
the \gls{acr:oqe} calculations in case of hazard models based on a 
logic-tree. 

All the tests described in this appendix are accessible at this
url \url{https://github.com/gem/oq-engine/tree/master/qa_tests/hazard/}.
%
\section{Theoretical background}
Assuming seismicity following a \textit{Poissonian} temporal occurrence model,
the calculation of the \textit{probability of (at least one) exceedance} of a
ground motion value x in a time span T, for an intensity measure IM, can be
computed by first calculating the annual rate of exceedance:
\begin{equation}
    \label{are}
    \lambda(IM \geq x) = \sum_{i=1}^{N_{sources}} \lambda_{i}(m\geq M_{min})
    \int_{M_{min}}^{M_{max}^{i}} \int_{r=0}^{\infty}
    P(IM\geq x | m,r)
    f_{i}(m)
    f_{i}(r)
    drdm
\end{equation}
where:
\begin{itemize}
\item $N_{sources}$: total number of sources in the source model
\item $\lambda_{i}(m\geq M_{min})$: annual rate of ruptures in the i-th source with magnitude greater than or equal to $M_{min}$
\item $M_{max}^{i}$: maximum magnitude in the i-th source
\item $P(IM\geq x | m,r)$: conditional probability for an intensity measure IM to exceede an intensity measure level (x), given magnitude
value (m) and distance (r)
\item $f_{i}(m)$: probability density function for magnitude, for the i-th source
\item $f_{r}(r)$: probability density function for distance, for the i-th source
\end{itemize}
and then computing the Poissonian probability of at least one exceedance using the equation:
\begin{equation}
    \label{poe}
    P(IM \geq x) = 1-e^{-\lambda(IM\geq x) T}
\end{equation}
Assuming ground motion distribution to follow (in log scale) a truncated normal 
distribution, we can compute $P(IM\geq x | m,r)$ as:
\begin{equation}
\label{cpoe}
P(IM\geq x | m,r) = 1 - \frac{\Phi(\frac{\ln(x) - \overline{\ln(IM;m,r)}} {\sigma}) - \Phi(-n_{trunc})}
					{\Phi(n_{trunc}) - \Phi(-n_{trunc})}
\end{equation}
where $\Phi$ is the normal cumulative distribution function. In the following
tests, a truncation level of 2 (that is $n_{trunc}=2$) will be used. In this
case, rounding to 4 digits, we obtain
$\Phi(2)=0.9772$, $\Phi(-2)=0.0228$, and $\Phi(2) - \Phi(-2)=0.9545$.
So we can rewrite equation \ref{cpoe} as: 
\label{cpoe2}
\begin{equation}
P(IM\geq x | m,r) = 1 - \frac{\Phi(\frac{\ln(x) - \overline{\ln(IM;m,r)}} {\sigma}) -0.0228}
					{0.9545}
\end{equation}
In case a truncation level of 0 is considered, we can write $P(IM\geq x | m,r)$ as:
\begin{equation}
\label{cpoe_0_trunc}
P(IM\geq x | m,r) = H(\overline{\ln(IM;m,r)} - \ln(x))
\end{equation}
where $H$ is the Heaviside (or step) function. That is $P(IM\geq x | m,r)=1$ if
$\overline{\ln(IM;m,r)} \geq \ln(x)$ and $0$
otherwise.
%...............................................................................
\subsection{GMPE}
The GMPE utilized for the calculation of the tests is \textcite{sadigh1997}, 
for PGA on rock sites, for strike slip events (rake = 0) with magnitude less or
equal to 6.5. The mean of the logarithm of PGA is predicted by the following
equation:
\begin{equation}
\label{meanPGA}
\overline{\ln(IM;m,r)} = -0.624 + m - 2.1 \ln(r + \exp(1.29649 + 0.25 m))
\end{equation}
where $r$ is the closest distance to the rupture. The standard deviation is given by:
\begin{equation}
\label{sigmaPGA}
\sigma = 1.39 - 0.14 m
\end{equation}
%...............................................................................
\subsection{Magnitude Scaling Relationship}
The magnitude scaling relationship utilized in the QA tests for calculating median rupture area from magnitude value is the
one defined in the PEER tests, that is:
\begin{equation}
\label{msr}
A = 10^{m - 4}
\end{equation}
where $A$ is the rupture area (in squared km).
%
%...............................................................................
\section{Tests} 
\section{Hazard curve calculation with different single source types}
\paragraph{Hazard curve calculation with single point
source, single magnitude MFD, and with ground motion truncation level = 2.0}
%
\textit{\textbf{NOTE}: This test is meant to exercise the hazard curve
calculator (both classical and event-based) with a point
source}\footnote{OQ-engine input files available at the following link
\url{https://github.com/gem/oq-engine/tree/master/qa_tests/hazard/classical/case_1}
and 
\url{https://github.com/gem/oq-engine/tree/master/qa_tests/hazard/event_based/case_1}}

Let's consider a source model consisting of a single point source, generating 
a single rupture of magnitude 4, with aspect ratio (length/width) equal to 1, 
vertical (dip=90.0), and with an annual occurrence rate ($\lambda(m=4)$) 
equal to 1. 
%
This means that the probability density function for magnitude is a dirac delta
function centered at m = 4, that is: \begin{equation} \label{t1pdfm} f(m) =
\delta(m - 4.0) \end{equation} Let's further assume that the rupture hypocenter
(located at the centroid of the rupture surface) is located at 4 km depth, and
let's assume upper seismogenic depth at 3.5 km, and lower seismogenic depth at
4.5 km.\\ Let's assume the site of interest to be right on top of the point
source (that is same latitude and longitude).\\ The rupture area (from eq.
\ref{msr}) is 1 squared km, and given that the aspect ratio is 1, the rupture
length and width are equal to 1 km. Given that the rupture hypocenter is assumed
to be in the centre of the rupture surface, the top of rupture depth is 3.5 km.
Given that the site is sitting just right on top of the point source, the
closest distance from site to rupture is 3.5 km.  Given that there is a single
rupture scenario, and this is the only possible distance value, the probability
density function for the closest distance to rupture is a dirac delta function
centered at r = 3.5, that is:
\begin{equation}
\label{t1pdfr}
f(r) = \delta(r - 3.5)
\end{equation}
Substituting equations \ref{t1pdfm} and \ref{t1pdfr} in equation \ref{are},
considering that there is only one source in the source model, and, assuming
$M_{min}=4$,  $\lambda_{i}(m\geq M_{min}) = \lambda(m=4) = 1$, equation
\ref{are} simplifies to:
\begin{equation}
\lambda(IM \geq x) = P(IM\geq x | 4.0, 3.5)
\end{equation}
The mean of the logarithm of PGA, as predicted by equation \ref{meanPGA} is
(rounded to 4 digits):
\begin{equation}
\overline{\ln(IM;m,r)} = -0.624 + 4.0 - 2.1\ln(3.5 + \exp(1.29649 + 0.25 * 4.0)) = -2.0802
\end{equation}
and the corresponding standard deviation (as from eq. \ref{sigmaPGA}) is:
\begin{equation}
\sigma = 1.39 - 0.14 * 4.0  = 0.83
\end{equation}
The rates of exceedance for PGA levels equal to 0.1, 0.4, 0.6 are:
\begin{eqnarray}
\lambda(PGA \geq 0.1) = 1 - \frac{\Phi(\frac{\ln(0.1) + 2.0802} {0.83}) -0.0228}
					{0.9545} = 0.6107 \nonumber \\
\lambda(PGA \geq 0.4) = 1 - \frac{\Phi(\frac{\ln(0.4) + 2.0802} {0.83}) -0.0228}
					{0.9545} = 0.0605 \nonumber \\
\lambda(PGA \geq 0.6) = 1 - \frac{\Phi(\frac{\ln(0.6) + 2.0802} {0.83}) -0.0228}
					{0.9545} = 0.0069 \nonumber \\
\end{eqnarray}
The corresponding probabilities of exceedance in a time period of 1 year ($T=1.0$) are:
\begin{eqnarray}
P(PGA \geq 0.1) = 1 - \exp(- 0.6107 * 1.0) = 0.4570 \nonumber \\
P(PGA \geq 0.4) = 1 - \exp(- 0.0605 * 1.0) = 0.0587 \nonumber \\
P(PGA \geq 0.6) = 1 - \exp(- 0.0069 * 1.0) = 0.0069 \nonumber \\
\end{eqnarray}
%
\clearpage
%
%...............................................................................
% Test 2
\paragraph{Hazard curve calculation with single point source, 
truncated GR MFD, and ground motion truncation level = 0}
\textit{\textbf{NOTE}: This test is meant to exercise the hazard curve
    calculator (both classical and event-based) with a point source
    having a Gutenberg Richter magnitude frequency distribution} \footnote{
    OQ-engine input files available at the following link
    \url{https://github.com/gem/oq-engine/tree/master/qa_tests/hazard/classical/case_2}
    and 
    \url{https://github.com/gem/oq-engine/tree/master/qa_tests/hazard/event_based/case_2}
}

This test assumes a source model consisting of a single point source, 
with a magnitude frequency distribution (MFD)
defined as a truncated Gutenbergh-Richter with the following parameters:
$a=2.0$, $b=1.0$, $M_{min}=4.0$, $M_{max}=7.0$.
The point source is assumed to generate ruptures at a single hypocentral depth
of 0.5 km, with an aspect ratio of 1.  The upper seismogenic depth is set to 0.0
(the lower seismogenic depth can be set to an arbitrary value, clearly larger
than the hypocentral depth).
%
Starting from the minimum magnitude all ruptures generated by the source will
reach the earth surface.  Indeed, for magnitude $M=4$, rupture area is 1 squared
km (from equation \ref{msr}) and assuming aspect ratio of 1 , rupture width is
1, and having set rupture hypocenter to 0.5 km, the rupture will reach the
surface. The site is right on top the source location, therefore for
all ruptures the closest distance to the site of interest is 0.0 km. 
%
This means that the probability density function for the closest distance to
rupture is a dirac delta function centered at r = 0.0, that is:
\begin{equation}
\label{t2pdfr}
f(r) = \delta(r)
\end{equation}
The probability density function for a truncated GR can be written as:
\begin{equation}
\label{tGRpdf}
f(m) = \frac{b \,\ln(10) 10^{-b(m-M_{min})}}{1 - 10^{-b(M_{max} - M_{min})}}
\end{equation}
Using a truncation level equal to 0, the probability of the logarithm of PGA
exceeding a level $x$ can be computed using equation \ref{cpoe_0_trunc}.
Substituting equations \ref{t2pdfr}, \ref{tGRpdf}, and \ref{cpoe_0_trunc} in
equation \ref{are}, and considering that the source model consists of only one
source, we can write:
\begin{eqnarray}
    \label{aret2}
    \lambda(PGA \geq x) =  \lambda(m\geq M_{min}) \int_{M_{min}}^{M_{max}}
    H(-0.624 + m - 2.1 (1.29649 + 0.25 m) - \ln(x)) \nonumber \\
    \frac{b \,\ln(10) 10^{-b(m-M_{min})}}{1 - 10^{-b(M_{max} - M_{min})}} dm =\nonumber \\
     \lambda(m\geq M_{min})\, \frac{b \,\ln(10)}{1 - 10^{-b(M_{max} - M_{min})}}
     \int_{M_{min}}^{M_{max}} H(0.475m - 3.346629 - \ln(x)) 10^{-b(m-M_{min})} dm =\nonumber \\
      \lambda(m\geq M_{min})\, \frac{b \,\ln(10)}{1 - 10^{-b(M_{max} -
      M_{min})}} \left[ -\frac{10^{-b(m - M_{min}})}{b\,\ln(10)} \right]_{max\{
      M_{min, \frac{3.346629 + \ln(x)}{0.475}} \}}^{M_{max}}
\end{eqnarray}
%
if $M_{max} \leq  \frac{3.346629 + \ln(x)}{0.475}$, otherwise the result of the
equation is 0.\\ if $M_{min} \geq  \frac{3.346629 + \ln(x)}{0.475}$, we can
simplify equation \ref{aret2} as:
%
\begin{equation}
\label{aret2simple}
\lambda(PGA \geq x) =  \frac{\lambda(m\geq M_{min})}{1 - 10^{-b(M_{max} - M_{min})}} (1 - 10^{-b(M_{max} - M_{min})})
\end{equation}
if $M_{min} \leq  \frac{3.346629 + \ln(x)}{0.475}$, we can write equation \ref{aret2} as:
\begin{equation}
\label{aret2simple2}
\lambda(PGA \geq x) =  \frac{\lambda(m\geq M_{min})}{1 - 10^{-b(M_{max} -
M_{min})}} (10^{-b( \frac{3.346629 + \ln(x)}{0.475} - M_{min})} - 10^{-b(M_{max} - M_{min})})
\end{equation}
%
The rate of exceedance for PGA level equal to 0.1 is (considering that
$(3.346629 +\ln(0.1)) / 0.475 \approx 2.2...$ and therefore using equation
\ref{aret2simple}):
%
\begin{equation}
\lambda(PGA \geq 0.1) =  \frac{10^{-2}}{1 - 10^{-3}} (1 - 10^{-3}) = 10^{-2}
\end{equation}
and the corresponding probability of exceedance in a time period of 1 year is (rounding to 5 digits):
\begin{equation}
P(PGA \geq 0.1) = 1 - \exp(- 10^{-2} * 1.0) = 0.00995
\end{equation}
The rate of exceedance for PGA level equal to 0.4 is (considering that
$(3.346629 +\ln(0.4)) / 0.475 = 5.11650$ and therefore using equation \ref{aret2simple2}):
\begin{equation}
\lambda(PGA \geq 0.4) =  \frac{10^{-2}}{1 - 10^{-3}} (10^{-(5.11650 - 4.0)} - 10^{-3}) = 0.00076
\end{equation}
and the corresponding probability of exceedance in a time period of 1 year is (rounding to 5 digits):
\begin{equation}
P(PGA \geq 0.4) = 1 - \exp(- 0.00076 * 1.0) = 0.00076
\end{equation}
The rate of exceedance for PGA level equal to 0.6 is (considering that
$(3.346629 +\ln(0.6)) / 0.475 = 5.97011$ and therefore using equation \ref{aret2simple2}):
\begin{equation}
\lambda(PGA \geq 0.6) =  \frac{10^{-2}}{1 - 10^{-3}} (10^{-(5.97011 - 4.0)} - 10^{-3}) = 9.7 \cdot 10^{-5}
\end{equation}
and the corresponding probability of exceedance in a time period of 1 year is (rounding to 5 digits):
\begin{equation}
P(PGA \geq 0.6) = 1 - \exp(- 9.7 \cdot 10^{-5} * 1.0) = 9.7 \cdot 10^{-5}
\end{equation}
The rate of exceedance for PGA level equal to 1.0 is 0, considering that  
$(3.346629 +\ln(1.0)) / 0.475 = 7.045531$ and therefore greater than the 
considered maximum magnitude and the corresponding probability of 
exceedance in a time period of 1 year is also 0:
\begin{equation}
P(PGA \geq 1.0) = 1 - \exp(- 0 * 1.0) = 0
\end{equation}
%
\clearpage
%
%...............................................................................
% test 3
\paragraph{Hazard curve calculation with single area source, single
magnitude MFD, and ground motion truncation level = 0} 
%
\textit{\textbf{NOTE}: This test is meant to exercise the hazard curve 
    calculator (classical) with an area source}\footnote{
    OQ-engine input files available at the following link
    \url{https://github.com/gem/oq-engine/tree/master/qa_tests/hazard/classical/case_3}}
 
This test assumes a source model consisting of a single area source, 
defined as circle of 5 km radius centered on the investigate site. 
%
The MFD contains a single magnitude ($M=4$) associated to an annual rate of
exceedance ($\lambda(m=4.0)$) equal to 1. The hypocentral depth of the area
source is set to 0.5 and the upper seismogenic depth to 0 km. Aspect ratio is
assumed equal to 1. Under this conditions all ruptures reach the surface. The
lower seismogenic depth can be set to a value greater or equal then 1.0 km. In
this case all ruptures have a square shape and have uniform length of 1 km. 

With these assumptions, the probability density function for magnitude is a
dirac delta function as given in equation \ref{t1pdfm}. 
%
Indicating with $l$ the rupture length and with $R$ the radius of the area
source, the probability density function for the closest distance to the rupture
is given by:
\begin{equation}
f(r) = 
\begin{cases}
\frac{2r}{R^{2}} + \frac{l}{2R^{2}} & \text{if } r \leq R - \frac{l}{2} \\
\frac{1}{R}         & \text{if } R - \frac{l}{2} < r \leq R \\
0                         &   \text{if } r > R \\
\end{cases}
\end{equation}
Assuming a truncation level equal to 0, we can write equation \ref{are} (considering a single source and $\lambda(m\geq M_{min}) = \lambda(m=4.0) = 1$) as:
\begin{eqnarray}
    \label{aret3}
    \lambda(IM \geq x) = \int_{0}^{R-\frac{l}{2}}\left[\frac{2r}{R^{2}} +
    \frac{l}{2R^{2}}\right] H(-0.624 + 4.0 - 2.1 \ln(r + \exp(1.29649 + 0.25 \cdot
    4.0)) \nonumber \\
    - \ln(x))dr + \frac{1}{R}\int_{R - \frac{l}{2}}^{R}H(-0.624 + 4.0 - 2.1 \ln(r
    + \exp(1.29649 + 0.25 \cdot 4.0)) - \ln(x))dr \nonumber \\
\end{eqnarray}
The Heaviside function is non-zero if and only if $-0.624 + 4.0 - 2.1 \ln(r +
\exp(1.29649 + 0.25 \cdot 4.0)) - \ln(x) \geq 0$, that is $r \leq e^{\frac{3.376
- \ln(x)}{2.1}} - e^{2.29649}$.
If $R\leq e^{\frac{3.376 - \ln(x)}{2.1}} - e^{2.29649}$, then we can solve equation \ref{aret3} as:
\begin{eqnarray}
\label{aret3case1}
\lambda(IM \geq x) = \frac{1}{R^{2}}(R - \frac{l}{2})^{2} + \frac{l}{2R^{2}}(R - \frac{l}{2}) + \frac{l}{2R}
\end{eqnarray}
if $R-\frac{l}{2}< e^{\frac{3.376 - \ln(x)}{2.1}} - e^{2.29649} < R$, then we can solve equation \ref{aret3} as:
\begin{eqnarray}
\label{aret3case2}
\lambda(IM \geq x) = \frac{1}{R^{2}}(R - \frac{l}{2})^{2} + \frac{l}{2R^{2}}(R -
\frac{l}{2}) + \frac{1}{R}(e^{\frac{3.376 - \ln(x)}{2.1}} - e^{2.29649} - R + \frac{l}{2})
\end{eqnarray}
if $0 < e^{\frac{3.376 - \ln(x)}{2.1}} - e^{2.29649} < R - \frac{l}{2} $, then we can solve equation \ref{aret3} as:
\begin{equation}
\label{aret3case3}
\lambda(IM \geq x) = \frac{1}{R^{2}} (e^{\frac{3.376 - \ln(x)}{2.1}} -
e^{2.29649})^{2} + \frac{l}{2R^{2}}(e^{\frac{3.376 - \ln(x)}{2.1}} - e^{2.29649})
\end{equation}
The rate of exceedance of PGA value equal to 0.1 is (considering that  $R = 5 <
e^{\frac{3.376 - \ln(0.1)}{2.1}} - e^{2.29649} = 5.00145$, and therefore using
equation \ref{aret3case1}) :
\begin{equation}
\lambda(IM \geq 0.1) = \frac{1}{5^{2}}(5 - 0.5)^{2} + \frac{1}{2\cdot5^{2}}(5 - 0.5) + \frac{1}{2\cdot5} = 1.0
\end{equation}
and the corresponding probability of exceedance in a time span of 1 year is:
\begin{equation}
P(IM \geq 0.1) = 1 - \exp(-1 \cdot 1) = 0.63212
\end{equation}
The rate of exceedance of PGA value equal to 0.12 is (considering that
$e^{\frac{3.376 - \ln(0.12)}{2.1}} - e^{2.29649} = 3.75902 < R - \frac{l}{2} = 4.5$ and therefore using equation \ref{aret3case3}):
\begin{equation}
\lambda(IM \geq 0.12) = \frac{1}{5^{2}} (e^{\frac{3.376 - \ln(0.12)}{2.1}} -
e^{2.29649})^{2} + \frac{1}{2\cdot 5^{2}}(e^{\frac{3.376 - \ln(0.12)}{2.1}} - e^{2.29649}) = 0.640389
\end{equation}
and the corresponding probability of exceedance in a time span of 1 year is:
\begin{equation}
P(IM \geq 0.12) = 1 - \exp(-0.640389 \cdot 1) = 0.47291
\end{equation}
The rate of exceedance of PGA value equal to 0.2 is (considering that
$e^{\frac{3.376 - \ln(0.2)}{2.1}} - e^{2.29649} = 0.80123 < R - \frac{l}{2} = 4.5$ and therefore using equation \ref{aret3case3}):
\begin{equation}
\lambda(IM \geq 0.2) = \frac{1}{5^{2}} (e^{\frac{3.376 - \ln(0.2)}{2.1}} -
e^{2.29649})^{2} + \frac{1}{2\cdot 5^{2}}(e^{\frac{3.376 - \ln(0.2)}{2.1}} - e^{2.29649}) = 0.04170
\end{equation}
and the corresponding probability of exceedance in a time span of 1 year is:
\begin{equation}
P(IM \geq 0.2) = 1 - \exp(-0.04170 \cdot 1) = 0.04084
\end{equation}
%
\clearpage
%
%...............................................................................
% test 4
\paragraph{Hazard curve calculation with single simple
fault source, single magnitude MFD, and ground motion truncation level = 0} 
\textit{\textbf{NOTE}:
This test is meant to exercise the hazard curve calculator (both classical and
event-based) with a simple fault source}\footnote{
    OQ-engine input files available at the following links:
    \url{https://github.com/gem/oq-engine/tree/master/qa_tests/hazard/classical/case_4}
    and 
    \url{https://github.com/gem/oq-engine/tree/master/qa_tests/hazard/event_based/case_4}}

This test assumes a source model
consisting of a single simple fault source, characterized by a single magnitude
MFD ($M=4$) with annual occurrence rate ($\lambda(m=4)$) equal to 1. The fault
source has an upper seismogenic depth of 0 km, and a lower seismogenic depth of
1 km.  Dip is 90 degrees. Aspect ratio is assumed 1.\\ Under these assumptions
the probability density function for magnitude is given by equation
\ref{t1pdfm}, while the probability density function for the closest distance
to the rupture (assuming the site to be in the middle point of the fault trace)
is:
\begin{equation}
f(r) = 
\begin{cases}
\frac{l}{L- l}\delta(r)    &    \text{if } r = 0 \\
\frac{2}{L-l}    &    \text{if } 0 < r \leq \frac{L}{2} - l \\
\end{cases}
\end{equation}
where $l$ is the rupture length, and $L$ is the fault length.\\
\todo[inline]{this is not clear. What's r?}
Assuming truncation level equal to 0, we can compute the rate of exceedance of
an intensity measure level as:
\begin{equation}
\lambda(PGA \geq x) = \frac{l}{L-l} + \frac{2}{L-l}min(\frac{L}{2} - l,
e^{\frac{3.376 - \ln(x)}{2.1}} - e^{2.29649})
\end{equation}
Assuming L = 10, and given that the rupture length $l=1$ (because aspect ratio is 1), for x = 0.1, the rate of exceedance is,
considering that $\frac{L}{2} - l = 4 < e^{\frac{3.376 - \ln(0.1)}{2.1}} - e^{2.29649} = 5.00145$:
\begin{equation}
\lambda(PGA \geq 0.1) = \frac{1}{9} + \frac{2}{9}4 = 1
\end{equation}
and therefore the probability of exceedance in 1 year is:
\begin{equation}
P(PGA \geq 0.1) = 1 - \exp(- 1\cdot 1) = 0.63212
\end{equation}
The rate of exceedance for x = 0.12 ($\frac{L}{2} - l = 4 > e^{\frac{3.376 - \ln(0.12)}{2.1}} - e^{2.29649} = 3.7590$) is:
\begin{equation}
\lambda(PGA \geq 0.12) = \frac{1}{9} + \frac{2}{9}3.7590 = 0.9464
\end{equation}
and the corresponding probability of exceedance in 1 year is:
\begin{equation}
P(PGA \geq 0.12) = 1 - \exp(- 0.9464 \cdot 1) = 0.61186
\end{equation}
The rate of exceedance for x = 0.2 ($\frac{L}{2} - l = 4 > e^{\frac{3.376 - \ln(0.2)}{2.1}} - e^{2.29649} = 0.801227$) is:
\begin{equation}
\lambda(PGA \geq 0.2) = \frac{1}{9} + \frac{2}{9}0.801227 = 0.28916
\end{equation}
and the corresponding probability of exceedance in 1 year is:
\begin{equation}
P(PGA \geq 0.2) = 1 - \exp(- 0.28916 \cdot 1) = 0.25110
\end{equation}
%
\clearpage
%
%...............................................................................
% test 5
\paragraph{Hazard curve calculation with single
complex fault source, single magnitude MFD, and ground motion truncation level =
0}
\textit{\textbf{NOTE}: This test is meant to exercise the hazard curve
calculator (both classical and event-based) with a complex fault source}\footnote{
    OQ-engine input files available at the following link
    \url{https://github.com/gem/oq-engine/tree/master/qa_tests/hazard/classical/case_5}} 

This test assumes a complex fault source composed of top and bottom edges. Both
the top and bottom edges have the same length, and same latitudes and
longitudes, but they are shifted in depth. 
Both the two edges consist of two line segments, one horizontal ($L_{1}$), and a
second , of length $L_{2}$, inclined of an angle $\alpha$ with respect to the
earth surface. The shift in depth between the top and bottom edges is of 1 km.

The source has a magnitude MFD characterized by a single magnitude
value equal to 4, and a corresponding annual rate $\lambda(m=4)$, equal to 1.
The aspect ratio is assumed 1. The rupture area is 1 squared km (as predicted
by equation \ref{msr}). Ruptures floating along the first section have a square
shape, but as soon as they start propagating along the inclined section, they
start acquiring the shape of a parallelogram.  
%
This means that while on the first section the rupture length is 1 km, on the
second section the rupture length is $l=\frac{A}{w\cdot sin(90 - \alpha)}$,
where A is the rupture area, $w$ is the fault width (that is the shift between
top and bottom edges), and $\alpha$ is the inclination of the second section
with respect to the horizontal direction.

The probability density function for magnitude is given by equation
\ref{t1pdfm}. Assuming a site located at the beginning of the first segment (and
assuming the first segment of the top edge starting at the earth surface), the
probability density function for the closest distance to the rupture is:
%
\begin{equation} f(r) = \begin{cases} \frac{1}{L_{1} + L_{2} - l}    &
\text{if } 0 \leq r \leq L_{1} \\ \frac{r}{(L_{1} + L_{2} -
l)\sqrt{L_{1}^{2}\cos^{2}\alpha - L_{1}^{2} + r^{2}}}                 &
    \text{if } L_{1} < r \leq r_{max} \\ \end{cases} \end{equation} where:
\begin{equation} 
    r_{max} = \sqrt{(L_{2} - l)^{2} + L_{1}^{2} + 2L_{1}(L_{2} - l)\cos\alpha} 
\end{equation} 
Assuming $L_{1}=3.0$ km, $L_{2}=3.0$ km,
$\alpha=30.0$ degrees, and $w=1$, the maximum closest distance to the rupture is
4.68973.\\ Assuming truncation level = 0, the rate of exceedance of a PGA level
x is (from equation \ref{are}): 
\begin{equation} \lambda(PGA \geq x) =
\int_{0}^{min(r_{max}, e^{\frac{3.376 - \ln(x)}{2.1}} - e^{2.29649})} f(r)dr
\end{equation} Considering x = 0.1, $r_{max} = 4.68973 < e^{\frac{3.376 -
\ln(0.1)}{2.1}}  - e^{2.29649} = 5.00145$, and therefore: 
\begin{eqnarray}
    \lambda(PGA \geq 0.1) = \frac{L_{1}}{L_{1}+L_{2}-l} +
    \frac{1}{L_{1}+L_{2}-l}\left[ \sqrt{L_{1}^{2}\cos^{2}\alpha-L_{1}^{2}+r^{2}}
    \right]_{L_{1}}^{r_{max}} =\nonumber \\ \frac{L_{1}}{L_{1}+L_{2}-l} +
    \frac{1}{L_{1}+L_{2}-l}\left[ \sqrt{L_{1}^{2}\cos^{2}\alpha - L_{1}^{2} +
    (L_{2}-l)^{2} + L_{1}^{2} + 2L_{1}(L_{2} - l)\cos\alpha} - L_{1}\cos\alpha
\right]=\nonumber \\ \frac{L_{1}}{L_{1}+L_{2}-l} + \frac{1}{L_{1}+L_{2}-l}\left[
\sqrt{(L_{1}\cos\alpha + (L_{2}-l))^{2}} - L_{1}\cos\alpha \right]=\nonumber \\
\frac{L_{1}}{L_{1}+L_{2}-l} + \frac{1}{L_{1}+L_{2}-l}\left[ L_{1}\cos\alpha
+L_{2} - l -L_{1}\cos\alpha \right] = 1 \nonumber \\
\end{eqnarray}
The corresponding probability of exceedance in 1 year is therefore:
\begin{equation}
P(PGA \geq 0.1) = 1 - \exp(-1\cdot 1) = 0.632120
\end{equation}
Considering x = 0.12, $r_{max} = 4.68973 > e^{\frac{3.376 - \ln(0.12)}{2.1}} - e^{2.29649} = 3.7590 > L_{1} = 3.0$, and therefore:
\begin{eqnarray}
\lambda(PGA \geq 0.12) = \frac{L_{1}}{L_{1}+L_{2}-l} +
\frac{1}{L_{1}+L_{2}-l}\left[ \sqrt{L_{1}^{2}\cos^{2}\alpha - L_{1}^{2} +
3.7590^{2}}-L_{1}\cos\alpha \right]=\nonumber \\
\frac{3}{3+3-\frac{2}{\sqrt{3}}} + \frac{1}{3+3-\frac{2}{\sqrt{3}}}\left[
\sqrt{9\cdot\frac{3}{4} - 9 + 3.7590^{2}} -\frac{3\sqrt{3}}{2}\right] = 0.79431
\nonumber \\
\end{eqnarray}
The corresponding probability of exceedance in 1 year is therefore:
\begin{equation}
P(PGA\geq0.12) = 1 -\exp(-0.79431 \cdot 1) = 0.54811
\end{equation}
Considering x = 0.2, $r_{max} = 4.68973 > e^{\frac{3.376 - \ln(0.2)}{2.1}} - e^{2.29649} = 0.801227 < L_{1} = 3.0$, and therefore:
\begin{equation}
\lambda(PGA \geq 0.2) = \frac{0.801227}{L_{1} + L_{2} - l} = \frac{0.801227}{3 + 3 - \frac{2}{\sqrt{3}}} = 0.16536
\end{equation}
The corresponding probability of exceedance in 1 year is therefore:
\begin{equation}
P(PGA \geq 0.2) = 1 - \exp(- 0.16536 \cdot 1) = 0.15241
\end{equation}
%
\clearpage
%
%...............................................................................
% test 6
\paragraph{Hazard curve calculation with source model consisting of
multiple sources} 

\textit{\textbf{NOTE}: This test is meant to exercise the
    parallelization strategy of the hazard curve calculator. 
    Given that the source model consists of multiple sources, a
    task for each source can be defined, and therefore the task creation and
    aggregation of the results can be tested.}\footnote{
    OQ-engine input files available at the following link
    \url{https://github.com/gem/oq-engine/tree/master/qa_tests/hazard/classical/case_6}}

This test assumes a source model
consisting of 2 seismic sources, as described in tests 4 and 5, that is a simple
fault source and a complex fault source.  For the simple fault source as
described in test 4, the rates of exceedance for PGA levels of 0.1, 0.12 and 0.2
are 1.0, 0.9464, 0.28916, respectively. For the complex fault source as
described in test 6, the rate of exceedance for the same PGA levels are: 1.0,
0.79431, 0.16536.\\ Using equation \ref{are} the rates of exceedance in case of
a source model consisting of the two above mentioned sources are: (1+1)=2.0,
(0.9464 + 0.79431) = 1.74071, (0.28916+0.16536) = 0.45452. The corresponding
probabilities of exceedance in a period of 1 year are:
\begin{eqnarray}
P(PGA \geq 0.10) &=& 1 - \exp(- 2 \cdot 1) = 0.86466 \nonumber \\
P(PGA \geq 0.12) &=& 1 - \exp(- 1.74071 \cdot 1) = 0.82460 \nonumber \\
P(PGA \geq 0.20) &=& 1 - \exp(- 0.45452 \cdot 1) = 0.36525 \nonumber \\
\end{eqnarray}
%
\clearpage
%
%...............................................................................
\section{Hazard curve calculation with logic-trees}
% test 7
\paragraph{Hazard curve calculation with logic tree containing
multiple source model} \textit{\textbf{NOTE}: This test is meant to exercise the
    hazard curve calculator (both classical and event-based) when considering a
    non-trivial logic tree (that is a logic tree with more than one path). The
    test should check that the correct solution for each path (when using Path
    Enumeration) is obtained and that the mean hazard curve is correctly
    computed. This test should also check that Monte Carlo Sampling and Path
    Enumeration should provide the same mean hazard curve.}\footnote{
    OQ-engine input files available at the following link
    \url{https://github.com/gem/oq-engine/tree/master/qa_tests/hazard/classical/case_7}}

This test assumes a logic tree defining 2 source models. Source model 1 consists
of a simple and a complex fault sources, as described in test 6, while source
model 2 consists of a single simple fault source as described in test 4.\\ For
PGA levels equal to 0.1, 0.12, 0.2, the probabilities of exceedance from source
model 1 are:
\begin{eqnarray}
P(PGA \geq 0.10) &=& 0.86466 \nonumber \\
P(PGA \geq 0.12) &=& 0.82460 \nonumber \\
P(PGA \geq 0.20) &=& 0.36525 \nonumber \\
\end{eqnarray}
the probabilities of exceedance from source model 2 are instead:
\begin{eqnarray}
P(PGA \geq 0.10) &=& 0.63212 \nonumber \\
P(PGA \geq 0.12) &=& 0.61186 \nonumber \\
P(PGA \geq 0.20) &=& 0.25110 \nonumber \\
\end{eqnarray}
Assuming source model 1 to be assigned to probability value of 0.7 and source model 2 to be assigned to probability value of 0.3, the mean hazard curve is:
\begin{eqnarray}
P(PGA \geq 0.10) &=& 0.7 * 0.86466 + 0.3 * 0.63212 = 0.794898\nonumber \\
P(PGA \geq 0.12) &=& 0.7 * 0.82460 + 0.3 * 0.61186 = 0.760778\nonumber \\
P(PGA \geq 0.20) &=& 0.7 * 0.36525 + 0.3 * 0.25110 = 0.331005\nonumber \\
\end{eqnarray}
\clearpage
%
\clearpage
%
%...............................................................................
% test 8
\paragraph{Hazard curve calculation with logic tree containing single
source model and a and b Gutenberg Richter absolute uncertainties}
\textit{\textbf{NOTE}: This test is meant to exercise the
'uncertaintyType="abGRAbsolute" ' option in the logic tree construction.}\footnote{
    OQ-engine input files available at the following link
    \url{https://github.com/gem/oq-engine/tree/master/qa_tests/hazard/classical/case_8}} 

This test assumes a logic tree defining a single source model. The source model
contains a single point source defining a Gutenberg Richter magnitude frequency
distribution, as defined in test 2. The logic tree defines absolute
uncertainties on the Gutenberg Richter a and b values:
\begin{eqnarray}
probability = 0.2, a = 2.2, b = 0.8 \nonumber \\
probability = 0.6, a = 2.0, b = 1.0 \nonumber \\
probability = 0.2, a = 1.8, b = 1.2 \nonumber \\
\end{eqnarray}
For the case $a=2.2$ and $b=0.8$, the rates of exceedance for PGA levels of 0.1, 0.4, 0.6, 1.0 are:
\begin{eqnarray}
\lambda(PGA \geq 0.1) &=& \frac{10^{2.2 - 0.8\cdot4.0}}{1 - 10^{-0.8(7.0 - 4.0)}}(1 - 10^{-0.8(7.0 - 4.0)}) = 0.1 \nonumber \\
\lambda(PGA \geq 0.4) &=& \frac{10^{2.2 - 0.8\cdot4.0}}{1 - 10^{-0.8(7.0 - 4.0)}}(10^{-0.8(5.11650-4.0)} - 10^{-0.8(7.0 - 4.0)}) = 0.012439 \nonumber \\
\lambda(PGA \geq 0.6) &=& \frac{10^{2.2 - 0.8\cdot4.0}}{1 - 10^{-0.8(7.0 - 4.0)}}(10^{-0.8(5.97011-4.0)} - 10^{-0.8(7.0 - 4.0)}) = 0.002265 \nonumber \\
\lambda(PGA \geq 1.0) &=& 0.0 \nonumber \\
\end{eqnarray}
The corresponding probabilities are:
\begin{eqnarray}
P(PGA \geq 0.1) &=& 0.095163\nonumber \\
P(PGA \geq 0.4) &=& 0.012362\nonumber \\
P(PGA \geq 0.6) &=& 0.002262\nonumber \\
P(PGA \geq 1.0) &=& 0.0\nonumber \\
\end{eqnarray}
For the case $a=2.0$, $b=1$, the rates of exceedance, (and the corresponding probabilities) are:
\begin{eqnarray}
\lambda(PGA \geq 0.1) &=& 10^{-2} \nonumber \\
\lambda(PGA \geq 0.4) &=& 0.00076 \nonumber \\
\lambda(PGA \geq 0.6) &=& 9.7\cdot 10^{-5} \nonumber \\
\lambda(PGA \geq 1.0) &=& 0.0 \nonumber \\
\end{eqnarray}
The corresponding probabilities are:
\begin{eqnarray}
P(PGA \geq 0.1) &=& 0.009950 \nonumber \\
P(PGA \geq 0.4) &=& 0.00076 \nonumber \\
P(PGA \geq 0.6) &=&  9.99995\cdot10^{-6} \nonumber \\
P(PGA \geq 1.0) &=&  0.0 \nonumber \\
\end{eqnarray}
For the case $a=1.8$ and $b=1.2$, the rates of exceedance are:
\begin{eqnarray}
\lambda(PGA \geq 0.1) &=& \frac{10^{1.8 - 1.2\cdot4.0}}{1 - 10^{-1.2(7.0 - 4.0)}}(1 - 10^{-1.2(7.0 - 4.0)})  = 0.001 \nonumber \\
\lambda(PGA \geq 0.4) &=& \frac{10^{1.8 - 1.2\cdot4.0}}{1 - 10^{-1.2(7.0 - 4.0)}}(10^{-1.2(5.11650-4.0)} - 10^{-1.2(7.0 - 4.0)}) = 4.5490\cdot 10^{-5} \nonumber \\
\lambda(PGA \geq 0.6) &=& \frac{10^{1.8 - 1.2\cdot4.0}}{1 - 10^{-1.2(7.0 - 4.0)}}(10^{-1.2(5.97011-4.0)} - 10^{-1.2(7.0 - 4.0)}) = 4.07366\cdot 10^{-6} \nonumber \\
\lambda(PGA \geq 1.0) &=& 0.0 \nonumber \\
\end{eqnarray}
The corresponding probabilities are:
\begin{eqnarray}
P(PGA \geq 0.1) &=& 0.0009995\nonumber \\
P(PGA \geq 0.4) &=& 4.5489\cdot 10^{-5}\nonumber \\
P(PGA \geq 0.6) &=& 4.07365\cdot 10^{-6}\nonumber \\
P(PGA \geq 1.0) &=& 0.0\nonumber \\
\end{eqnarray}
%
\clearpage
%
%...............................................................................
% test 9
\paragraph{Hazard curve calculation with logic tree containing single
source model and absolute uncertainties on Gutenberg Richter Maximum Magnitude}
%
\textit{\textbf{NOTE}: This test is meant to exercise the
'uncertaintyType="maxMagGRAbsolute" ' option in the logic tree construction}\footnote{
    OQ-engine input files available at the following link
    \url{https://github.com/gem/oq-engine/tree/master/qa_tests/hazard/classical/case_9}
    }

This test assumes a logic tree defining a single source model. The source model
contains a single point source defining a Gutenberg Richter magnitude frequency
distribution, as defined in test 2. The logic tree defines absolute
uncertainties on the Gutenberg Richter maximum magnitude:
\begin{eqnarray}
probability = 0.5, Mmax = 7.0 \nonumber \\
probability = 0.5, Mmax = 7.5 \nonumber \\
\end{eqnarray}
For the case $Mmax = 7.0$, the probabilities of exceedance are:
\begin{eqnarray}
P(PGA \geq 0.1) &=& 0.00995 \nonumber \\
P(PGA \geq 0.4) &=& 0.00076 \nonumber \\
P(PGA \geq 0.6) &=& 9.7 \cdot 10^{-5} \nonumber \\
P(PGA \geq 1.0) &=& 0
\end{eqnarray}
For the case $Mmax = 7.5$, the rates of exceedance are:
\begin{eqnarray}
\lambda(PGA \geq 0.1) &=&  \frac{10^{-2}}{1 - 10^{-3.5}} (1 - 10^{-3.5}) = 0.01\nonumber \\ 
\lambda(PGA \geq 0.4) &=&  \frac{10^{-2}}{1 - 10^{-3.5}} (10^{-(5.11650 - 4.0)} - 10^{-3.5}) = 0.000762\nonumber \\
\lambda(PGA \geq 0.6) &=&  \frac{10^{-2}}{1 - 10^{-3.5}} (10^{-(5.97011 - 4.0)} - 10^{-3.5}) = 0.000104\nonumber \\ 
\lambda(PGA \geq 1.0) &=&  \frac{10^{-2}}{1 - 10^{-3.5}} (10^{-(7.5 - 4.0)} - 10^{-3.5}) = 0.0
\end{eqnarray}
The corresponding probabilities of exceedance in 1 year are:
\begin{eqnarray}
P(PGA \geq 0.1) &=&  0.00995\nonumber \\
P(PGA \geq 0.4) &=&  0.00076\nonumber \\
P(PGA \geq 0.6) &=&  0.000104\nonumber \\
P(PGA \geq 1.0) &=& 0.0
\end{eqnarray}
%
\clearpage
%
%...............................................................................
% Test 10
\paragraph{Hazard curve calculation with logic tree containing source
model and relative uncertainties on Gutenberg Richter b value}
\textit{\textbf{NOTE}: This test is meant to exercise the
'uncertaintyType="bGRRelative" ' option in the logic tree construction}\footnote{
    OQ-engine input files available at the following link
    \url{https://github.com/gem/oq-engine/tree/master/qa_tests/hazard/classical/case_10}}

This
test assumes a logic tree defining a single source model. The source model
contains a single point source defining a Gutenberg Richter magnitude frequency
distribution, as defined in test 2. The logic tree defines relative
uncertainties on the Gutenberg Richter b value:
\begin{eqnarray}
probability = 0.5 &-& \delta b = 0.0 \nonumber \\
probability = 0.5 &-& \delta b = +0.4 \nonumber \\
\end{eqnarray}
For the case $\delta b = 0.0$, the probabilities of exceedance are:
\begin{eqnarray}
P(PGA \geq 0.1) &=& 0.00995 \nonumber \\
P(PGA \geq 0.4) &=& 0.00076 \nonumber \\
P(PGA \geq 0.6) &=& 9.7 \cdot 10^{-5} \nonumber \\
P(PGA \geq 1.0) &=& 0
\end{eqnarray}
For the case $\delta b = +0.4$, the a value of the Gutenberg Richter magnitude frequency distribution is changed to conserve the total moment rate. The
new a value is 4.243. The rates of exceedance are:
\begin{eqnarray}
\lambda(PGA \geq 0.1) &=&  \frac{10^{4.243 - 1.4 \cdot 4.0}}{1 - 10^{-1.4(7 - 4)}} (1 - 10^{-1.4(7 - 4)}) = 0.04395\nonumber \\ 
\lambda(PGA \geq 0.4) &=&  \frac{10^{4.243 - 1.4 \cdot 4.0}}{1 - 10^{-1.4(7 - 4)}} (10^{-1.4(5.11650 - 4.0)} - 10^{-1.4(7 - 4)}) = 0.0012\nonumber \\
\lambda(PGA \geq 0.6) &=&  \frac{10^{4.243 - 1.4 \cdot 4.0}}{1 - 10^{-1.4(7 - 4)}} (10^{-1.4(5.97011 - 4.0)} - 10^{-1.4(7 - 4)}) = 7.394 \cdot 10^{-5}\nonumber \\ 
\lambda(PGA \geq 1.0) &=&  0.0 
\end{eqnarray}
The corresponding probabilities of exceedance in 1 year are:
\begin{eqnarray}
P(PGA \geq 0.1) &=&  0.043\nonumber \\
P(PGA \geq 0.4) &=&  0.0012\nonumber \\
P(PGA \geq 0.6) &=&  7.394 \cdot 10^{-5}\nonumber \\
P(PGA \geq 1.0) &=& 0.0
\end{eqnarray}
%
\clearpage
%
%...............................................................................
% test 11
\paragraph{Hazard curve calculation with logic tree containing single
source model and relative uncertainties on Gutenberg Richter Maximum Magnitude}
\textit{\textbf{NOTE}: This test is meant to exercise the
    'uncertaintyType="maxMagGRRelative" ' option in the logic tree construction.
    It also allows to check the calculation of mean and quantile hazard
curves.}\footnote{
    OQ-engine input files available at the following link
    \url{https://github.com/gem/oq-engine/tree/master/qa_tests/hazard/classical/case_11}}

This test assumes a logic tree defining a single source model. The
source model contains a single point source defining a Gutenberg Richter
magnitude frequency distribution, as defined in test 2. The logic tree defines
relative uncertainties on the Gutenberg Richter maximum magnitude:
\begin{eqnarray}
probability = 0.2 & - & \delta M = +0.5 \nonumber \\
probability = 0.6 & - & \delta M = 0.0 \nonumber \\
probability = 0.2 & - & \delta M = -0.5 \nonumber \\
\end{eqnarray}
For the case $\delta M = +0.5$, the new a value is 1.7438. The corresponding rates of exceedance for PGA levels of 0.1, 0.4, 0.6, 1.0 are:
\begin{eqnarray}
\lambda(PGA \geq 0.1) &=& \frac{10^{1.7438 - 1.0\cdot4.0}}{1 - 10^{-1.0(7.5 - 4.0)}}(1 - 10^{-1.0(7.5 - 4.0)}) = 0.0055\nonumber \\
\lambda(PGA \geq 0.4) &=& \frac{10^{1.7438 - 1.0\cdot4.0}}{1 - 10^{-1.0(7.5 - 4.0)}}(10^{-1.0(5.11650-4.0)} - 10^{-1.0(7.5 - 4.0)}) = 0.00042\nonumber \\
\lambda(PGA \geq 0.6) &=& \frac{10^{1.7438 - 1.0\cdot4.0}}{1 - 10^{-1.0(7.5 - 4.0)}}(10^{-1.0(5.97011-4.0)} - 10^{-1.0(7.5 - 4.0)}) =  5.77\cdot 10^{-5}\nonumber \\
\lambda(PGA \geq 1.0) &=& 0.0
\end{eqnarray}
The corresponding probabilities are:
\begin{eqnarray}
P(PGA \geq 0.1) &=& 0.0055\nonumber \\
P(PGA \geq 0.4) &=& 0.00042\nonumber \\
P(PGA \geq 0.6) &=& 5.77\cdot 10^{-5}\nonumber \\
P(PGA \geq 1.0) &=& 0.0\nonumber \\
\end{eqnarray}
For the case $\delta M = 0.0$, the probabilities of exceedance are equal to the one provided in test 2, that is:
\begin{eqnarray}
P(PGA \geq 0.1) &=& 0.00995 \nonumber \\
P(PGA \geq 0.4) &=& 0.00076 \nonumber \\
P(PGA \geq 0.6) &=& 9.7 \cdot 10^{-5} \nonumber \\
P(PGA \geq 1.0) &=&  0
\end{eqnarray}
For the case $\delta M = -0.5$, the new a value is 2.261. The corresponding rates of exceedance for PGA levels of 0.1, 0.4, 0.6, 1.0 are:
\begin{eqnarray}
\lambda(PGA \geq 0.1) &=& \frac{10^{2.261 - 1.0\cdot4.0}}{1 - 10^{-1.0(6.5 - 4.0)}}(1 - 10^{-1.0(6.5 - 4.0)}) = 0.018\nonumber \\
\lambda(PGA \geq 0.4) &=& \frac{10^{2.261 - 1.0\cdot4.0}}{1 - 10^{-1.0(6.5 - 4.0)}}(10^{-1.0(5.11650-4.0)} - 10^{-1.0(6.5 - 4.0)}) = 0.0013\nonumber \\
\lambda(PGA \geq 0.6) &=& \frac{10^{2.261 - 1.0\cdot4.0}}{1 - 10^{-1.0(6.5 - 4.0)}}(10^{-1.0(5.97011-4.0)} - 10^{-1.0(6.5 - 4.0)}) =  0.00014\nonumber \\
\lambda(PGA \geq 1.0) &=& 0.0
\end{eqnarray}
The corresponding probabilities are:
\begin{eqnarray}
P(PGA \geq 0.1) &=& 0.018\nonumber \\
P(PGA \geq 0.4) &=& 0.0013\nonumber \\
P(PGA \geq 0.6) &=& 0.00014\nonumber \\
P(PGA \geq 1.0) &=& 0.0\nonumber \\
\end{eqnarray}
The mean probabilies of exceedance are:
\begin{eqnarray}
P(PGA \geq 0.1) &=& 0.2 * 0.0055 + 0.6 * 0.00995 + 0.2 * 0.018 = 0.01067\nonumber \\
P(PGA \geq 0.4) &=& 0.2 * 0.00042 + 0.6 * 0.00076 + 0.2 * 0.0013 = 0.0008\nonumber \\
P(PGA \geq 0.6) &=& 0.2 * 5.77\cdot 10^{-5} + 0.6 * 9.7 \cdot 10^{-5} + 0.2 * 0.00014 = 9.774e-05\nonumber \\
P(PGA \geq 1.0) &=& 0.2 * 0.0 + 0.6 * 0.0 + 0.2 * 0.0 = 0.0\nonumber \\
\end{eqnarray}
The probabilities of exceedance for quantile level 0.1 are:
\begin{eqnarray}
P(PGA \geq 0.1) &=& 0.0055\nonumber \\
P(PGA \geq 0.4) &=& 0.00042\nonumber \\
P(PGA \geq 0.6) &=& 5.77\cdot 10^{-5}\nonumber \\
P(PGA \geq 1.0) &=& 0.0\nonumber \\
\end{eqnarray}
The probabilities of exceedance for quantile level 0.9 are:
\begin{eqnarray}
P(PGA \geq 0.1) &=& \frac{0.018 -  0.00995}{1.0 - 0.8}(0.9 - 0.8) + 0.00995 = 0.013975\nonumber \\
P(PGA \geq 0.4) &=& \frac{0.0013 -  0.00076}{1.0 - 0.8}(0.9 - 0.8) + 0.00076 = 0.00103\nonumber \\
P(PGA \geq 0.6) &=& \frac{0.00014 -  9.7\cdot 10^{-5}}{1.0 - 0.8}(0.9 - 0.8) + 9.7\cdot 10^{-5} = 0.0001185\nonumber \\
P(PGA \geq 1.0) &=& 0.0
\end{eqnarray}
%
\clearpage
%
%...............................................................................
% 12
\section{Other tests}
\paragraph{Hazard curve calculation with source model consisting of
multiple sources belonging to different tectonic region types and therefore
requiring different GMPEs} 
\label{sec:test12}
\textit{\textbf{NOTE}: This test is meant to check
that GMPEs are correctly associated to each source based on the tectonic region
type.}\footnote{
    OQ-engine input files available at the following link
    \url{https://github.com/gem/oq-engine/tree/master/qa_tests/hazard/classical/case_12}
    and 
    \url{https://github.com/gem/oq-engine/tree/master/qa_tests/hazard/event_based/case_12}
} 

This test considers a source model consisting of two point sources. The
first point source is the one described in test 1, and the associated rates of
exceedance (using \textcite{sadigh1997} and truncation level equal to 2)
are:
 \begin{eqnarray}
\lambda(PGA \geq 0.1) & = & 0.6107 \nonumber \\
\lambda(PGA \geq 0.4) & = & 0.0605 \nonumber \\
\lambda(PGA \geq 0.6) & = & 0.0069 \nonumber \\
\end{eqnarray}
The second source is again a point source generating a single rupture of
magnitude 4.5 associated to an annual occurrence rate equal to 1. The rupture
aspect ratio is assumed equal to 1, and the area predicted by the Peer scaling
relationship is therefore $10^{0.5}$. The rupture length and width are equal to
$10^{0.25}$. The rupture is assumed vertical ($dip = 90.0$) and the hypocenter
is set at $\frac{10^{0.25}}{2}$, so that the rupture reaches the surface.  This
rupture is associated to the \textcite{boore2008} GMPE, which uses $R_{JB}$.
Assuming the site of interest to be on the same location of the point source,
$R_{JB} = 0$.\\
Assuming $m = 4.5$, $R_{JB} = 0$, $Vs_{30} = 760.0$, and $rake = 0$ (that is strike slip event), the median PGA is:
\begin{equation}
\overline{\ln(PGA;m=4.5,r=0)} = - 1.86841
\end{equation}
and the total standard deviation is:
\begin{equation}
\sigma = 0.564
\end{equation}
The rates of exceedance for PGA levels equal to 0.1, 0.4, 0.6 are:
\begin{eqnarray}
\lambda(PGA \geq 0.1) & = & 1 - \frac{\Phi(\frac{\ln(0.1) + 1.86841} {0.564}) -0.0228}
					{0.9545} = 0.79266 \nonumber \\
\lambda(PGA \geq 0.4) & = & 1 - \frac{\Phi(\frac{\ln(0.4) + 1.86841} {0.564}) -0.0228}
					{0.9545} = 0.02409 \nonumber \\
\lambda(PGA \geq 0.6) & = & 0.0\nonumber \\
\end{eqnarray}
The rate of exceedance for PGA = 0.6 is zero because $\frac{\ln(0.6) + 1.86841} {0.564} = 2.4$ which is larger than the truncation level, that is 2.0. 
The corresponding probabilities of exceedance in a time period of 1 year ($T=1.0$) considering the two sources are:
\begin{eqnarray}
P(PGA \geq 0.1) & = & 1 - \exp(- (0.6107 + 0.79266) * 1.0) =  0.75423\nonumber \\
P(PGA \geq 0.4) & = & 1 - \exp(- (0.0605 + 0.02409) * 1.0) =  0.08111\nonumber \\
P(PGA \geq 0.6) & = & 1 - \exp(- (0.0069 + 0.0) * 1.0) = 0.00688 \nonumber \\
\end{eqnarray}
%
\clearpage
%
%...............................................................................
\paragraph{Hazard curve calculation (event based approach) with ground
motion correlation included} 
%
\textit{\textbf{NOTE}: This test is meant to check hazard curve calculation on a
single site with ground motion correlation - that is using event based
approach.}\footnote{
OQ-engine input files available at the following link
\url{https://github.com/gem/oq-engine/tree/master/qa_tests/hazard/classical/case_13} 
and 
\url{https://github.com/gem/oq-engine/tree/master/qa_tests/hazard/event_based/case_13} 
}

This test consider a source model consisting of a single point source as 
defined in test at page \pageref{sec:test12} (point source with $m = 4.5$).
The GMPE used is \textcite{boore2008}
%
The test computes hazard using an event based approach; ground motion fields
are computed considering spatial correlation \parencite{jayaram2009}

The rates of exceedance predicted by the source model are:
\begin{eqnarray}
\lambda(PGA \geq 0.1) & = &  0.79266 \nonumber \\
\lambda(PGA \geq 0.4) & = &  0.02409 \nonumber \\
\lambda(PGA \geq 0.6) & = & 0.0 \nonumber \\
\end{eqnarray}
and the corresponding probabilities of exceedance are:
\begin{eqnarray}
P(PGA \geq 0.1) & = & 1 - \exp(- 0.79266 * 1.0) = 0.54736\nonumber \\
P(PGA \geq 0.4) & = & 1 - \exp(- 0.02409 * 1.0) =  0.02380\nonumber \\
P(PGA \geq 0.6) & = & 1 - \exp(- 0.0 * 1.0) = 0 \nonumber \\
\end{eqnarray}

%\section{Test 12 - Hazard curve calculation with logic tree containing multiple
%source models and uncertainties applying only to specific source models}
%\textit{\textbf{NOTE}: This test is meant to exercise the 'applyToBranches'
%option in the logic tree construction}\\

%\section{Test 13 - Hazard curve calculation with logic tree containing a single
%source model consisting of multiple sources and uncertainties applying to a
%specific source} \textit{\textbf{NOTE}: This test is meant to exercise the
%''applyToSources" option in the logic tree construction}\\

%\section{Test 14 - Hazard curve calculation with logic tree containing a single
%source model consisting of multiple sources belonging to different tectonic
%region types and uncertainties applying only to sources belonging to a specific
%tectonic region type} \textit{\textbf{NOTE}: This test is meant to exercise the
%"applyToTectonicRegionType" option in the logic tree construction}\\

%\section{Test 15 - Hazard curve calculation with logic tree containing a single
%source model consisting of multiple sources of different type and uncertainties
%applying only to sources belonging to a specific source type}
%\textit{\textbf{NOTE}: This test is meant to exercise the "applyToSourceType"
%option in the logic tree construction}\\

%\section{Test 16 - Hazard curve calculation with logic tree containing source
%model containing a single source model consisting of multiple sources belonging
%to different tectonic region types, and multiple GMPEs each associated to a
%different tectonic region type} \textit{\textbf{NOTE}: This test is meant to
%check that different GMPEs are correctly associated to different sources based
%on their tectonic region type}\\
\cleardoublepage
