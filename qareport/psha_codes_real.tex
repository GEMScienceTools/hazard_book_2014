In this chapter we present comparisons between seismic hazard results computed
using OpenQuake-engine and alternative software for real PSHA. We present three
cases: the seismic hazard model for the 2005 national building code of Canada
(\cite{adams2003}, \cite{halchuk2008}), the 2008 U.S. national seismic hazard
model (\cite{petersen2008}), and the seismic hazard model for a nuclear power
plant in South Africa (\cite{bommer2013}). We refer the reader to the original
reports for an in-depth description of each model. Here we present only the
main features and the aspects of interest for the software implementation.
%
% ..............................................................................
\section{The 2005 Canada national seismic hazard model}
%
% ..............................................................................
\subsection{The seismic source model}
To capture epistemic uncertainties in the seismic source definition, two
complete seismic source models are defined in the Canada national seismic hazard
model: the historical (\textbf{H}) and regional (\textbf{R}) models. The
historical model uses relatively small source zones drawn around historical
seismicity clusters, while the regional model defines larger, regional zones
reflecting seismotectonic units. Both models are composed of area sources, with
the only exception of the Queen Charlotte fault (in western Canada) modelled as
a fault source. 

Occurrence rates for each source are defined through a double-truncated
Gutenberg-Richter distribution, with minimum magnitude equal to 4.75. Epistemic
uncertainties in the magnitude-frequency distribution are captured by the
definition for each source of three possible ($a_{GR}$, $b_{GR}$) pairs and
three possible maximum magnitudes. Each source is also associated to three
possible hypocentral depths.

The model prescribes also a floor model (\textbf{F}) for the relatively aseismic
central part of Canada and a deterministic model for the Cascadia subdution zone
(\textbf{C}). Earthquake ruptures associated with area sources are assumed to
have no spatial extension (that is point ruptures), while earthquakes on faults
follow a magnitude-length scaling relationship ($L=10^{-1.085 + 0.389 m}$).
%
% ..............................................................................
\subsection{The ground motion model}
The ground motion model distinguishes between eastern and western Canada because
of the different properties in the crust. For eastern and central Canada the
GMPE model of \citet{ab1995} is used. For western Canada the model of
\citet{bjf1993} is used for shallow crustal sources, while for deep intraslab
sources the model of \citet{y1997} is adopted. Epistemic uncertainties are
included by defining, for each GMPE, a pair of parallel alternative relations,
with higher and lower mean values.
%
% ..............................................................................
\subsection{Reference site conditions}
Hazard maps are computed for a \textit{reference} ground condition corresponding
to "Site Class C" (firm-ground), defined by a 360 to 750 m/s $Vs_{30}$. For
central and eastern Canada, hazard map values computed with the hard-rock GMPE
of \citet{ab1995} are adjusted for firm-ground. That is, seismic hazard spectral
values are amplified by a period-dependent 'reference ground condition' factor
(see Table 2 in \cite{adams2003}). For western Canada, the model of
\citet{y1997} is adjusted for firm-ground. The model of \citet{bjf1993} does not
require instead any adjustment given that its "Soil Class B" is identical to
"Site Class C".
%
% ..............................................................................
\subsection{Implementation of the model in the OpenQuake-engine}
Currently, only the \textbf{H} and \textbf{R} models are implemented in the
OpenQuake-engine. We set the discretization step for area sources to 5 km, and
model ruptures as points. The discretization step for the Queen Charlotte fault
is instead 2 km, and the rupture extension is modeled using the magnitude-area
scaling relationship of \citet{wells1994}. The OpenQuake-engine does not
currently support the inclusion of magnitude-length scaling relationships and
this prevent us from exactly reproducing the original rupture modeling for the
Queen Charlotte fault.

For each source in both the \textbf{H} and \textbf{R} model, we computed a mean
magnitude-frequency distribution by considering all possible (that is 9)
($a_{GR}$, $b_{GR}$) - $M_{max}$ combinations. That is, for each magnitude bin
(of 0.1 magnitude units width), we defined the occurrence rate as the weighted
mean of the rates obtained from the different possible magnitude-frequency
distributions.

For each site, contributions from ruptures that are within a radius of 600 km
are considered for the central and eastern Canada models and of 400 km for the
western Canada models (consistently with what reported by \cite{adams2003}). The
original calculation assumes the ground motion distribution to be untruncated.
To reproduce such condition, we assume in the OpenQuake-engine calculation a
truncation level equal to 6 sigmas.
%
% ..............................................................................
\subsection{Comparison against Canada hazard maps}
We compare the OpenQuake-engine results against hazard maps obtained from
\textit{mean} hazard curves produced by the Geological Survey of Canada (GSC)
using a modified version of the commercial software FRISK88
(http://www.riskeng.com/software/frisk88m/). The GSC approach for constructing
hazard maps (for a given probability of exceedance or return period) relies on
the so-called 'robust' method (\cite{adams2003}). The method is based on
choosing, from the four models (\textbf{H}, \textbf{R}, \textbf{F} and
\textbf{C}), the highest value for each grid point accross Canada.

Using the OpenQuake-engine we thus computed hazard map values for the sites
across Canada for which the \textbf{H} or \textbf{R} models give the highest
values. A comparison for the $10\%$ probability of exceedance map for PGA is
shown in Figure \ref{fig:canada_475y_hmaps}.
\begin{figure}
\centering
\includegraphics[width=14cm]{./qareport/pictures/GSC_combined_PGA_0pt1_firm_ground.pdf}
\includegraphics[width=14cm]{./qareport/pictures/OQ_combined_PGA_0pt1_firm_ground.pdf}
\caption{Official hazard map produced by the Geological Survey of Canada (top) and by the OpenQuake-engine implementation (bottom)}
\label{fig:canada_475y_hmaps}
\end{figure}
\begin{figure}
\centering
\includegraphics[width=14cm]{./qareport/pictures/GSC_OQ_PGA_0pt1_abs_diff.pdf}
\includegraphics[width=14cm]{./qareport/pictures/GSC_OQ_PGA_0pt1_percent_diff.pdf}
\caption{Absolute (top) and percent (bottom) difference maps between the official GSC hazard map and the one produced by the OpenQuake-engine as visible in Figure \ref{fig:canada_475y_hmaps}}
\label{fig:canada_475y_dmaps}
\end{figure}
The comparison between the hazard maps reveals a satisfactory agreement, at
least from a visual perspective. Difference maps (Figure
\ref{fig:canada_475y_dmaps}) are however more informative and provide a
quantitative understanding of the differences between the two maps.

The absolute difference map shows that for the vast majority of the grid points,
the GSC solution over-estimates the OpenQuake-engine solution by values between
0 and 0.04 g. The highest positive differences (GSC > OpenQuake-engine), up to
0.3 g, are instead only visible along the Queen Charlotte fault, especially at
the north and south endings of the fault trace. The highest negative differences
(GSC < OpenQuake-engine) are still along the Queen Charlotte fault, especially
in the region close to the middle part of the fault trace. These differences can
be reasonably motivated by the previously mentioned fact that, for the Queen
Charlotte fault, we cannot use in the OpenQuake-engine a magnitude-length
scaling relationship as originally defined by the model. We thus used the
\citet{wells1994} magnitude-area scaling relationship. The discrepancies reflect
not only different functional forms but also different ways of defining
ruptures' size and shape. Indeed when using a magnitude-length scaling
relationship, no rupture reshaping (for area conservation) is required which is
instead a key feature when considering a magnitude-area scaling law.

The relative percent error shows instead the significance of the absolute
difference in the various regions covered by the hazard map. Percent differences
ranges from $-20\%$ to $30\%$. The highest positive differences are visible
along the area sources associated with low hazard values. Some significant
differences are visible along the boundaries of rectangular zones such as the
small rectangular area source close to the village of Anna, in Ohio, U.S. (85W,
40N) and the source south of the Newfoundland, Canada (approximately 60W, 45N).
Inside those sources the hazard levels are relatively high (between 0.15 and 0.2
g for the former and between 0.3 and 0.4 for the latter). Within the boundary
the relative error is less than $5\%$. The larger differences along the boundary
may be therefore the effect of different discretization algorithms in the two
software.
%
% ..............................................................................
\section{The 2008 U.S. national seismic hazard model}
%
% ..............................................................................
\subsection{The seismic source model}
To reproduce earthquake occurrences in various tectonic settings, the seismic
source model defines different typologies of seismic sources:
gridded-seismicity, uniform seismicity zones, and fault sources.

Gridded-seismicity and source zones are used to model seismicity occurring off
known faults, and moderate-size earthquakes not included in fault sources.
Occurrence rates are defined through a Gutenberg-Richter magnitude frequency
distribution, with minimum magnitude equal to 5. Earthquakes smaller than $M=6$
are treated as point ruptures, while for larger events finite ruptures are
generated. Rupture lengths are determined by using the magnitude-length scaling
relationship of \citet{wells1994}. Uncertainties in rupture strike are taken
into account by assigning to each site an average distance from ruptures with
strike directions uniformly distributed from $0^{\circ}$ to $180^{\circ}$.
Dipping ruptures are also modeled in the western U.S. by including an average
hanging-wall effect.

For the most part, fault sources are defined to model earthquakes with magnitude
larger than 6.5. Alternative fault source models are defined to take into
account different magnitude-frequency distributions: Characterisitic
\citep{schwartscoppersmith1984} and Gutenberg-Richter. Alternative fault
geometries are also included.

The source model distinguishes between central and eastern U.S. (CEUS) and
western U.S. (WUS). In CEUS the maximum magnitude is distributed between 6.6 and
7.2 for cratonic regions, and between 7.1 and 7.7 for the extended margin. Four
alternative gridded-seismicity models are defined: three that take into account
different completeness levels and one that considers uniform source zones for
the Eastern Tennessee and New Madrid seismic zones. A number of uniform
background zones are also defined to provide a hazard floor in areas with little
or no historical seismicity. Fault sources are defined for the New Madrid
seismic zone, and for the Meers fault (Oklahoma) and Cheraw fault (Colorado).
For New Madrid, five possible fault traces are defined and three possible return
periods (500-year, 750-year, 1000-year) are considered. Moreover, a temporal
clustering model is also included which considers the possible occurrence of
three dependent events. For the Charleston (South Carolina) seismic zone, two
source zones are defined in connection with the $M=7.3$ event occurred in 1886.

In WUS, the maximum magnitude for gridded (shallow) seismicity is 7.0 in most
regions. Exceptions include areas close to modeled faults. If a fault follows a
Gutenberg-Richter magnitude frequency distribution, then the maximum magnitude
in the neighbouring region is set to 6.5 (which is the $M_{min}$ of the fault
Gutenberg-Richter relation). If a fault follows a Characteristic model, then the
maximum magnitude is set to the minimum between 7.0 and the fault characteristic
magnitude. For deep seismicity, the maximum magnitude is instead set to 7.2. For
shallow seismicity, a single gridded-seismicity model is defined which, together
with a number of uniform background zones and special zones, represent the
overall distributed seismicity in WUS. Fault sources are defined for specific
regions: the Intermountain West, Pacific Northwest (including Cascadia
subduction faults), and California. Earthquake occurrence rates are both
described by a Characteristic and Gutenberg-Richter distribution. Epistemic
uncertainties of $\pm 0.2$ magnitude units are defined for both the
Characteristic earthquake magnitude and the maximum magnitude of the
Gutenberg-Richter relation. In addition to epistemic uncertainties, aleatory
variability is included for characteristic earthquakes using a normal
distribution with standard deviation of 0.12.
%
% ..............................................................................
\subsection{The ground motion model}
The ground motion model specifies different sets of GMPEs for CEUS and WUS. For
CEUS, a set of seven GMPEs is defined to represent the epistemic uncertainties
in the ground motion modeling: \citet{frankel1996}, \citet{somerville2001},
\citet{campbell2003SCR}, \citet{toro1997}, \citet{atkinson2006},
\citet{tavakoli2005}, \citet{silva2002}. For active shallow crust in WUS, the
GMPEs used are \citet{boore2008}, \citet{campbell2008}, and \citet{chiou2008}.
For subduction interface events (Cascadia) the GMPE models adopted are:
\citet{y1997}, \citet{ab2003}, \citet{zhao2006}. For subduction intraslab the
GMPEs of \citet{geomatrix1993} and \citet{ab2003} are used instead. For active
shallow crust in WUS, an additional level of epistemic uncertainties is included
to take into account data limitations especially for large earthquakes. Indeed,
for each GMPE, two symmetrical relationships are obtained by adding/subtracting
a scaling factor to the median value.
%
% ..............................................................................
\subsection{Reference site conditions}
Hazard maps are computed for a reference site condition corresponding to the
'NEHRP B/C site' condition which is associated to a $V_{s30}$ value equal to
760.0 m/s. The GMPEs used for active shallow crust in WUS allows for a direct
usage of the $V_{s30}$ value. For subduction GMPEs the 'rock' site class is
selected instead. Some of the CEUS GMPEs do not support the B/C site conditions.
Therefore a 'kappa' correction is applied to convert CEUS GMPEs from hard rock
to firm rock condition.
%
\begin{figure}
\centering
\includegraphics[width=14cm]{./qareport/pictures/map_usa_PGA_0pt1_NSHMP.pdf}
\includegraphics[width=14cm]{./qareport/pictures/map_usa_PGA_0pt1_OQ.pdf}
\caption{Official hazard map produced by the USGS-NSHMP (top) and by the OpenQuake-engine implementation (bottom)}
\label{fig:usa_475y_hmaps}
\end{figure}
\begin{figure}
\centering
\includegraphics[width=14cm]{./qareport/pictures/map_usa_PGA_0pt1_abs_diff_NSHMP_OQ.pdf}
\includegraphics[width=14cm]{./qareport/pictures/map_usa_PGA_0pt1_percent_diff_NSHMP_OQ.pdf}
\caption{Absolute (top) and percent (bottom) difference maps between the
official USGS-NSHMP hazard map and the one produced by the OpenQuake-engine as
visible in Figure \ref{fig:usa_475y_hmaps}}
\label{fig:usa_475y_dmaps}
\end{figure}
%
% ..............................................................................
\subsection{Implementation of the model in the OpenQuake-engine}
In the OpenQuake-engine implementation, gridded-seismicity models and uniform
seismicity zones are defined as collections of point sources. Point source
properties reflect model's specification. If the model defines uncertain
ruptures strike, then each point source is associated to multiple strike values
(between $0^{\circ}$ and $180^{\circ}$ for vertical ruptures and between
$0^{\circ}$ and $360^{\circ}$ for dipping ruptures) otherwise a single strike is
defined. For 'shallow' gridded seismicity models, the hypocentral depth is set
to 7.5 km (equal to the depth of the centroid of dipping rupture planes
(\cite{petersen2008})) and the lower seismogenic depth is set to 15 km. Deep
seismicity is assumed to occur at an hypocentral depth of 50 km, and the
seismogenic layer is assumed to extend between 50 and 100 km. Crustal faults are
defined as simple fault sources (with discretization step of 4 km), while
Cascadia subduction interface faults are defined as complex fault sources (with
discretization stepf of 10 km). The magnitude-area scaling relationship of
\cite{wells1994} is used for all models (gridded-seismicity and fault based)
both in WUS and CEUS. In the current OQ-engine implementation, all source models
are included except the cluster model for the New Madrid seismic zone (only the
unclustered source model is considered). To combine models from different logic
tree branches, we compute for each source a magnitude-frequency distribution
whose occurrence rates are scaled by the branch weight. For each site,
contributions coming from ruptures that are within 1000 km are considered for
the CEUS and Cascadia models, and 200 km for the gridded-seismicity and the
shallow crustal faults models in WUS. Ground motion distribution is truncated at
three sigmas (\cite{petersen2008}). Epistemic uncertainties on GMPE median
values for WUS active shallow crust are not yet implemented.
%
% ..............................................................................
\subsection{Comparison against USA hazard map}
The comparison between PGA hazard maps for $10\%$ probability of exceedance in
50 years produced by the USGS-NSHMP and by the OpenQuake-implementation is shown
in Figure \ref{fig:usa_475y_hmaps}. The visual comparison shows an overall
agreement. A more quantitative view on the differences between the two maps is
given in Figure \ref{fig:usa_475y_dmaps}. Absolute differences range from -0.07
g to 0.12 g. For the vast majority of the sites, differences range from -0.01 g
0.02 g. Close to WUS crustal faults (except California) and in the Charlevoix
(Quebec, Canada) zone the OQ-engine provides larger values. On the contrary,
close to some southern California crustal faults and close to the New Madrid
zone the OQ-engine provides lower values than the USGS-NSHMP. This is also
visible in the percent difference map, where in a broad area around the New
Madrid seismic zone differences range mostly from $0$ to $10\%$ with peak values
up to $15\%$. Lower values in the OpenQuake-engine solutions are also visible
close to the Charleston zone and in Oregon/Washington states (WUS).

Differences in the New Madrid zone can be accounted for by the absence, in the
OQ-engine implementation, of the clustered source model. Discrepancies in the
WUS can be instead explained as due to the missing epistemic uncertainties in
the GMPE median value for shallow crustal sources. In addition to that,
differences can arise because of the different scaling relationship used. In
California, the model of \cite{hanks2002} is used, while in the OpenQuake-engine
calculation the model of \cite{wells1994} is adopted.
%
% ..............................................................................
\section{PSHA for the Thyspunt nuclear site in South Africa}
\citet{bommer2013} describes the use of the OpenQuake-engine for validating
the logic-tree implementation in a PSHA carried out for the coastal site of
Thyspunt, on the southern coast of South Africa, a location selected for the
possible construction of a new nuclear power plant.

The PSHA, done following the guidelines of a SSHAC level 3 project
(\cite{budnitz1997}), required the calculation procedure, and in particular the
implementation of the logic tree representing the full set of uncertainties, to
be subject to Quality Assurance (QA). To provide QA checks, the logic tree was
simultaneously implemented in two different, and entirely independent, software
packages.

The FRISK88 software was selected to perform the official seismic hazard
calculations while the OpenQuake-engine was used as alternative software to
verify the correctness of the logic tree implementation. The seismic source
model consisted of five fault sources and five area sources. 
%
Epistemic uncertainties were defined by considering alternative zone boundaries,
recurrence parameters, fault-ruptures characteristics and maximum magnitude
values. The final logic tree comprised more than 1000 branch combinations for
each area and more than 800 for each fault. 
%
The ground motion model was based on adjusted forms of the GMPEs proposed by 
\citet{abrahamson2008}, \citet{chiou2008}, and \citet{akkar_cagnan_2010} which
in combination of various models for the sigma led to a total of 216 branches in
the ground motion model logic tree.  Considering both the source and ground
motion model logic trees, the number of PSHA input combinations was of the order
of 2 millions.
%
\begin{figure}
\centering
\includegraphics[width=10cm]{./qareport/pictures/ThyspunktSourceModel.jpg}
\caption{Seismic source model (area zones and faults) defined for the Thyspunkt site.}
\label{fig:thyspunkt_source_model}
\end{figure}
%
Verification of the logic tree implementation was carried out by comparing mean
hazard curves for each seismic source considering the full seismic source model
logic tree, but only a single branch in the ground motion model logic tree. In
order to increase the chance of identifying possible errors, all the three GMPE
models were used and three different periods were selected. 
%
Hazard curves computed by the two software for fault and area sources are
presented in Figures \ref{fig:thyspunkt_curves_fault} and
\ref{fig:thyspunkt_curves_area} respectively.

The level of agreement is in general very high considering the complexity in the
seismic source model logic tree. For the two fault sources closest to the site,
PLET and GAM the two pairs of hazard curves are almost identical. For the area
sources, the agreement is generally even better, and neither of the hazard codes
produces consistently higher or lower results.
%
\begin{figure}
\centering
\includegraphics[width=10cm]{./qareport/pictures/ThyspunktCurvesFault.jpg}
\caption{Mean hazard curves obtained for the fault sources using FRISK88 (solid
curves) and the OpenQuake-engine (dashed curves)}
\label{fig:thyspunkt_curves_fault}
\end{figure}
%
\begin{figure}
\centering
\includegraphics[width=10cm]{./qareport/pictures/ThyspunktCurvesArea.jpg}
\caption{Mean hazard curves obtained for the area sources using FRISK88 (solid curves) and the OpenQuake-engine (dashed curves)}
\label{fig:thyspunkt_curves_area}
\end{figure}
