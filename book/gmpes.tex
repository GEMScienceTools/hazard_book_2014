This chapter provides an overview of ground shaking intensity modeling based 
on empirical equations and describes the way \glspl{acr:gsim} (more commonly 
known as ground motion models or \glspl{acr:gmpe}) are implemented in the 
\gls{acr:oqe}.
%
% ..............................................................................
\section{Introduction}
%
\Glspl{groundshakingintensitymodel} are empirical equations that - given a 
set of parameters - compute a value representative of the shaking 
intensity (or any other effect produced by an earthquake such as fault 
displacement) together with an associated uncertainty. 
%
\gls{acr:gsim} have a fundamental importance in the overall \gls{acr:psha} 
architecture.

A ground shaking intensity equation can be schematised as follows 
\parencite{alatik2010}: 
\begin{equation}
Y = f(X_{es},\theta)+\Delta
\end{equation}
where $Y$ is the natural logarithm of a ground shaking intensity measure, 
$X_{es}$ is the vector of explanatory (or independent) variables, $\theta$ 
is the vector of model coefficients and $\Delta$ is a random variable 
describing the overall variability of the ground shaking intensity at 
the site.

The selection of independent variables and the definition of the structure 
of the equation is usually done on the basis of physical principles and 
basic descriptions of the earthquake process, the latter
intended as the combination of a rupture occurrence, the synchronous 
radiation of seismic waves and their propagation to the site.
%
% . . . . . . . . . . . . . . . . . . . . . . . . . . . . . . . . . . . . . . .
\subsection{Main predictor variables}
\index{Ground-motion prediction equation!Predictor variables}
In the current section we give a brief overview of the most important predictor 
variables \cite[see][for a summary]{akkar2013r}supported by the \gls{acr:oqe}; 
currently they are organised into three main groups: variables describing the 
rupture properties, variables describing the rupture\--site path and variables 
used to characterise the site conditions. 
%
Table \ref{tab:parameters} contains a summary of the variables
assigned to the three groups.
% --------------------------------------------------------------------->>> Table
\begin{table}[!h]
\centering
\begin{tabular}{|p{5cm}p{8cm}|}
\hline
\rowcolor{anti-flashwhite}
\bf{Group} & \bf{Variables} \\
\hline 
Rupture parameters & - Magnitude\\
                   & - Dip \\ 
                   & - Z$_{tor}$ \\ 
                   & - Rake \\ \hline
Rupture-site distances & - See Table \ref{tab:distances} \\ \hline
Site parameters & - V$_{S,30}$ \\
                & - Depth to the 1 km/s interface interface \\
                & - Depth to the 2.5 km/s interface \\ 
\hline
\end{tabular}
\caption{Principal predictor variables supported by the \gls{acr:oqe} and 
    corresponding groups.}
\label{tab:parameters}
\end{table}
% ---------------------------------------------------------------------<<< Table
%
\subsubsection{Magnitude}
\index{Ground-motion prediction equation!predictor variables!Magnitude}
Moment magnitude \parencite{hanks1979} is the magnitude typology 
preponderantly used within the most recent \gls{acr:gsim} and as a 
consequence within seismic hazard analysis in general. 
%
However, in the \gls{acr:oqe} there is not a predefined magnitude typology.
It is up to the user to ensure that the magnitude used to define earthquake 
occurrence in the \gls{seismicsourcemodel} is consistent with the one used 
in the selected \glspl{groundshakingintensitymodel}.
%
\subsubsection{Distance}
\index{Ground-motion prediction equation!predictor variables!Source-site 
distance}
%
The OpenQuake-engine supports almost all the rupture-site 
distance metrics used by the most recent and complex 
\glspl{groundshakingintensitymodel} published in the scientific 
literature (see \ref{tab:distances} for a comprehensive list).
% --------------------------------------------------------------------->>> Table
\begin{table}[!b]
\centering
\begin{tabular}{p{5cm}cp{5cm}}
\hline
\rowcolor{anti-flashwhite}
\bf{Distance definition} & \bf{Symbol} & \bf{Description} \\
\hline 
Epicentral & R$_{epi}$ & \\
Hypocentral & R$_{hypo}$ & \\
Joyner and Boore distance & R$_{jb}$ & Closest distance from the surface 
    projection of the rupture \\
Rupture distance & R$_{rup}$ & Closest distance to the rupture surface \\
Horizontal top-edge distance & R$_{x}$ & Horizontal distance from the top 
    edge of the rupture \\
Top-of-Rupture depth & Z$_{tor}$ & Depth to the top edge of the rupture \\
\hline
\end{tabular}
\caption{Rupture-site distances supported by the \gls{acr:oqe}.}
\label{tab:distances}
\end{table}
% ---------------------------------------------------------------------<<< Table
The calculation of distances within the hazard component of the \gls{acr:oqe} 
is performed by assuming a spherical earth with a radius of 6371.0 km. 
%
\subsubsection{Rupture mechanism}
\index{Ground-motion prediction equation!Rupture mechanism}
Many \glspl{acr:gsim} compute ground-motion according to 
different categories identifying major rupturing mechanisms such as normal,
strike-slip or reverse.

In the \gls{acr:oqe} the rupture mechanism of a seismic source is specified 
in terms of the rake angle (defined according to \cite{aki2002}). 
%
Since the rake is not used directly as a predictor variable in 
\glspl{acr:gsim}, most of the \gls{acr:oqe} implementations contain a mapping
between the rake angle and the rupture mechanism classes supported by each
specific model.
%
\subsubsection{Time averaged shear-wave velocity in the uppermost 30m 
(V$_{S,30}$)}
\index{Ground-motion prediction equation!Rupture mechanism! Time averaged
shear-wave velocity in the uppermost 30m, V$_{S,30}$}
%
Local soil conditions and their effects on the ground-motion are routinely 
incorporated into \glspl{groundshakingintensitymodel} by means of a scalar
quantity corresponding to the time-averaged shear wave velocity measured 
in the uppermost 30m of the soil column (V$_{S,30}$).
%
Local soil conditions in the \gls{acr:oqe} are specified by means of this
parameter.

In case of \glspl{groundshakingintensitymodel} which support the definition 
of local soil conditions through soil classes (e.g. hard rock, soft soil) 
their implementation is done in a way that given a value of V$_{S,30}$ the
corresponding soil class is used to compute the value of ground motion 
(provided that a mapping between soil classes and V$_{S,30}$ is defined 
by the authors).

Additional parameters used to quantitatively describe local geology are 
the depths to the 1 km/s and 2.5 km/s shear-wave velocity interfaces. 
These are parameters used in some \glspl{acr:gsim} (e.g. \cite{chiou2008}) 
to quantify the overall thickness of the soil column. 
%
\subsubsection{Depth to the top-of-rupture (Z$_{tor}$)}
\index{Ground-motion prediction equation!Rupture mechanism!Depth to the top of
rupture}
The depth to the top of rupture is a parameter introduced in some of the NGA
West 1 \gls{acr:gsim} such as \textcite{chiou2008} and \textcite{abrahamson2008}
following a supposed dependence of ground-motion to the depth of the source,
as suggested by \textcite{somerville2006a}.
%
% ..............................................................................
\section{Use of GMPEs in seismic hazard analysis}
% 
The standard process adopted for the implementation into the \gls{acr:oqe} 
of a \gls{groundshakingintensitymodel} requires a 
set of verification tables which contain values of ground-motion (or standard 
deviation) computed using a large number of the possible predictor 
variables combination. 

Through these tables and an automated verification procedure, it is 
possible to achieve with a high confidence level the consistency between 
the original results and the corresponding values computed with the 
version of the \gls{acr:gsim} implemented.
%
% . . . . . . . . . . . . . . . . . . . . . . . . . . . . . . . . . . . . . . .
\subsection{Testing}
%
The progressively increasing complexity of \glspl{groundshakingintensitymodel}
is giving more and more emphasis and relevance to the validation between 
between the results provided by the \gls{acr:gsim} implemented in
\gls{acr:psha} codes and the results of original implementations 
described in the scientific literature or directly provided by the authors.

The \gls{acr:oqe} provides an automated procedure for the testing of 
the implemented \glspl{groundshakingintensitymodel}. 
%
% . . . . . . . . . . . . . . . . . . . . . . . . . . . . . . . . . . . . . . .
\subsection{Selection criteria}
The number of published \gls{acr:gsim} increased exponentially over the last
few years.  
%
\subsubsection{Pre-selection criteria}
\cite{cotton2006} 

%
\subsubsection{Selection criteria}
%
%
% ..............................................................................
\section{Calculation of ground motion fields}
%
The calculation of \glspl{groundmotionfield} is an essential step in the
calculation of hazard via an \gls{epsha} and for the calculation of losses 
in case of spatially distributed assets.
% . . . . . . . . . . . . . . . . . . . . . . . . . . . . . . . . . . . . . . .
\subsection{Spatial correlation of ground motion}
%
% ..............................................................................
\section{Modified GMPEs}
%
% . . . . . . . . . . . . . . . . . . . . . . . . . . . . . . . . . . . . . . .
\subsection{Adjusted equations}
%
% . . . . . . . . . . . . . . . . . . . . . . . . . . . . . . . . . . . . . . .
\subsection{Non-ergodic sigma}
%
% ..............................................................................
\section{Future developments}
%
% . . . . . . . . . . . . . . . . . . . . . . . . . . . . . . . . . . . . . . .
\subsection{Sigma adjustment}
%
% . . . . . . . . . . . . . . . . . . . . . . . . . . . . . . . . . . . . . . .
\subsection{Vs-Kappa correction}
%
% . . . . . . . . . . . . . . . . . . . . . . . . . . . . . . . . . . . . . . .
\subsection{Host-to-target adjustment}
%
% . . . . . . . . . . . . . . . . . . . . . . . . . . . . . . . . . . . . . . .
\subsection{Spatial cross correlation}
%
% . . . . . . . . . . . . . . . . . . . . . . . . . . . . . . . . . . . . . . .
\subsection{Near source directivity effects}

