This chapter provides an overview of ground shaking intensity modeling based 
empirical equations and describes the way \gls{acr:gsim} (more commonly 
known as ground motion models or \glspl{acr:gmpe}) are implemented in the 
\gls{acr:oqe}.
%
% ..............................................................................
\section{Introduction}
%
\glspl{groundshakingintensitymodel} are empirical equations that - given a 
small set of parameters 
- compute a value representative value of the shaking intensity (or any other 
effect produced by an earthquake such as fault displacement) together with 
an associated uncertainty. 
%
\gls{acr:gsim} have a fundamental importance in the overall \gls{acr:psha} 
architecture.

A ground shaking intensity equation can be schematised as follows 
\parencite{alatik2010}: 
\begin{equation}
Y = f(X_{es},\theta)+\Delta
\end{equation}
where $Y$ is the natural logarithm of a ground shaking intensity measure, 
$X_{es}$ is the vector of explanatory (or independent) variables, $\theta$ 
is the vector of model coefficients and $\Delta$ is a random variable 
describing the overall variability of the ground shaking intensity at 
the site.
%
The selection of independent variables and the definition of the structure 
of the equation is usually done on the basis of physical principles and 
basic descriptions of the earthquake process, the latter
intended as the combination of a rupturing event and the synchronous 
radiation of seismic waves and their propagation to a site.
%
% . . . . . . . . . . . . . . . . . . . . . . . . . . . . . . . . . . . . . . .
\subsection{Main predictor variables}
\index{Ground-motion prediction equation!Predictor variables}
%
\subsubsection{Magnitude}
Moment magnitude \parencite{hanks1979} is the magnitude typology 
preponderantly used today. 
%
In the \gls{acr:oqe} however there is not a predefined magnitude typology.
It is up to the user to ensure that the magnitude used to define earthquake 
occurrence in the \gls{seismicsourcemodel} is consistent with the one used 
in the selected \glspl{groundshakingintensitymodel}.
%
%
% . . . . . . . . . . . . . . . . . . . . . . . . . . . . . . . . . . . . . . .
\subsubsection{Distance}
\index{Ground-motion prediction equation!Source-site distances}
%
The OpenQuake-engine supports almost all the most common rupture-site 
distance metrics used by the most recent and complex 
\glspl{groundshakingintensitymodel}. 
%
\begin{table}[!b]
\centering
\begin{tabular}{p{5cm}cp{5cm}}
\hline
\rowcolor{anti-flashwhite}
\bf{Distance definition} & \bf{Symbol} & \bf{Description} \\
\hline 
Epicentral & R$_{epi}$ & \\
Hypocentral & R$_{hypo}$ & \\
Joyner and Boore distance & R$_{jb}$ & Closest distance from the surface 
    projection of the rupture \\
Rupture distance & R$_{rup}$ & Closest distance to the rupture surface \\
Horizontal top-edge distance & R$_{x}$ & Horizontal distance from the top 
    edge of the rupture \\
Top-edge depth & Z$_{tor}$ & Depth to the top edge of the rupture \\
\hline
\end{tabular}
\caption{Rupture-site distances supported by the \gls{acr:oqe}.}
\end{table}
%
\paragraph{Calculation of distances}
\index{Ground-motion prediction equation!Calculation of distances}
%
% . . . . . . . . . . . . . . . . . . . . . . . . . . . . . . . . . . . . . . .
\subsubsection{Rupture mechanism}
\index{Ground-motion prediction equation!Rupture mechanism}
%
\subsubsection{V$_{S,30}$}
\index{Ground-motion prediction equation!Rupture mechanism}
%
Local soil conditions and their effects on the ground motion are incorporated
into \glspl{groundshakingintensitymodel} by means of the time-averaged shear
wave velocity measured in the shallowest 30m of material below the topographic
surface. This is also the parameter used by \gls{acr:oqe}.

In case of \glspl{groundshakingintensitymodel} which support the definition 
of local soil conditions through classes (e.g. hard rock, soft soil) their
implementation is done in a way that given a value of V$_{S,30}$ the
corresponding soil class is used to compute the value of ground motion 
(provided that a mapping between soil classes and V$_{S,30}$ was defined 
by the authors).
%
% ..............................................................................
\section{Use of GMPEs in seismic hazard analysis}
% 
Despite their crucial role in the calculation of seismic hazard the process
currently adopted for the development and dissemination of
\glspl{groundshakingintensitymodel} do not fulfill the strict requirements
needed to guarantee a completely consistent implementation of these models into
the \gls{acr:oqe}.
%
% . . . . . . . . . . . . . . . . . . . . . . . . . . . . . . . . . . . . . . .
\subsection{Testing}
%
The progressively increasing complexity of \glspl{groundshakingintensitymodel}
is giving more and more emphasis and relevance to the validation between 
between the results provided by the \gls{acr:gsim} implemented in
\gls{acr:psha} codes and the results of original implementations 
described in the scientific literature or directly provided by the authors.

The \gls{acr:oqe} provides an automated procedure for the testing of 
the implemented \glspl{groundshakingintensitymodel}. 
%
% . . . . . . . . . . . . . . . . . . . . . . . . . . . . . . . . . . . . . . .
\subsection{Selection criteria}
The number of published \gls{acr:gsim} increased exponentially over the last
few years.  
%
\subsubsection{Pre-selection criteria}
\cite{cotton2006} 

%
\subsubsection{Selection criteria}
%
%
% ..............................................................................
\section{Calculation of ground motion fields}
%
The calculation of \glspl{groundmotionfield} is an essential step in the
calculation of hazard via an \gls{epsha} and for the calculation of losses 
in case of spatially distributed assets.
% . . . . . . . . . . . . . . . . . . . . . . . . . . . . . . . . . . . . . . .
\subsection{Spatial correlation of ground motion}
%
% ..............................................................................
\section{Modified GMPEs}
%
% . . . . . . . . . . . . . . . . . . . . . . . . . . . . . . . . . . . . . . .
\subsection{Adjusted equations}
%
% . . . . . . . . . . . . . . . . . . . . . . . . . . . . . . . . . . . . . . .
\subsection{Non-ergodic sigma}
%
% ..............................................................................
\section{Future developments}
%
% . . . . . . . . . . . . . . . . . . . . . . . . . . . . . . . . . . . . . . .
\subsection{Sigma adjustment}
%
% . . . . . . . . . . . . . . . . . . . . . . . . . . . . . . . . . . . . . . .
\subsection{Vs-Kappa correction}
%
% . . . . . . . . . . . . . . . . . . . . . . . . . . . . . . . . . . . . . . .
\subsection{Host-to-target adjustment}
%
% . . . . . . . . . . . . . . . . . . . . . . . . . . . . . . . . . . . . . . .
\subsection{Spatial cross correlation}
%
% . . . . . . . . . . . . . . . . . . . . . . . . . . . . . . . . . . . . . . .
\subsection{Near source directivity effects}
