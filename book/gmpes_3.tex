%
% ..............................................................................
\section{Future developments}
%
Strong ground-motion seismology is an actively evolving discipline where
new advancements are constantly introduced.  
%
The current implementation of GSIMs in the \gls{acr:oqe} already offers advanced
features although - according to our experience and the feedback we received
from the \gls{acr:oqe} users it partially lacks of flexibility and should offer
a less difficult process for the incorporating new \glspl{acr:gsim} perhaps
though tables. 

A revision of the current scheme adopted for the implementation of new 
\glspl{acr:gsim} will be therefore necessary in the future in order to add
new features. 
%
An incomplete list of the possible improvements that might be introduced 
is the following one:
%
\begin{itemize}
\item Support for host-to-target adjustment \hfill \\
    Within site specific hazard analyses \citep[see for example][]{bommer2014}
    as well as within regional hazard studies in areas with scarce strong-motion
    recordings it is common to adjust GMPEs in order to properly take into
    account regional variations of parameters controlling ground-motion
    properties such as stress drop, kappa and average shear wave velocity within
    the uppermost 30 meters.
\item Sigma adjustment, support for homoskedastic sigma \hfill \\ 

\item Spatial cross correlation \hfill \\
    
\item Near source directivity effects \hfill \\
\end{itemize}
