%
% ..............................................................................
\section{Future developments}
%
The current implementation of GSIMs into the \gls{acr:oqe} offers advanced
features but - according to experience and feedback received
from users - it partially lacks of flexibility and should provide
an easier process for the incorporation of new \glspl{acr:gsim}. 
%
A revision of the current scheme adopted for the implementation of new 
\glspl{acr:gsim} will be therefore necessary in order to 
address the suggestions and feedback we received so far and add new features.
 
An incomplete list of the possible improvements that might be introduced 
is the following one:
%
\begin{itemize}
    \item Better support for \glspl{acr:gsim} defined via tables \hfill \\
        This is a request we received from different experts. Technically it is
        already possible to create \gls{acr:gsim} from tables (see for example 
        the OQ-engine implementation of the \textcite{frankel1996} available
        on
        \href{https://github.com/gem/oq-hazardlib/blob/master/openquake/hazardlib/gsim/frankel_1996.py}{github}
        \footnote{
            \href{https://github.com/gem/oq-hazardlib/blob/master/openquake/hazardlib/gsim/frankel_1996.py}
        {https://github.com/gem/oq-hazardlib/blob/master/openquake/hazardlib/gsim/frankel\_1996.py}})
        but better support with an illustration of the process to be 
        followed will be necessary.
    \item Support for host-to-target adjustment \hfill \\
        %
        Within site specific hazard analyses \citep[see for
        example][]{bommer2014} as well as within regional hazard studies in
        areas with scarce strong-motion recordings it is common to adjust GMPEs
        in order to properly take into account regional variations of parameters
        controlling ground-motion properties such as stress drop, kappa and
        average shear wave velocity within the uppermost 30 meters. 
        
        In future releases of the OpenQuake-engine we plan to improve the
        support for these modifications while maintaining a high as possible 
        the level and quality of testing.
        %
    \item Sigma adjustment, support for homoskedastic sigma \hfill \\ 
        %
        These methods are also commonly adopted within site-specific
        hazard analyses.
        %
        These corrections are currently supported by the OQ-engine by 
        subclassing a prototype GSIM implementation. A subclass is a copy of
        an original class; it inherits properties of the original class. Its 
        behavior can be modified by adding new components or by overriding 
        the existing ones. 
        %
        However, this requires programming experience. We therefore plan to 
        offer easier procedures for using these methods with the \gls{acr:gsim}
        implemented.
        %
    \item Spatial cross correlation \hfill \\
        %
        The \gls{acr:oqe} already supports the calculation of ground motion
        fields generated by taking into account the spatial correlation of
        within event ground motion residuals.
        %
        We plan to add the possibility of computing cross-correlated ground 
        motion fields in order to better support analyses taking into account 
        distributed infrastructures as well as heterogeneous portfolios of 
        assets. 
    \item Near source directivity effects \hfill \\
        %
        Some of the recently published NGA West 2 GMPEs (i.e.
        \textcite{chiou2014}) offer the possibility of computing ground motion
        by taking into account near source directivity effects. We plan to
        implement this GMPE - as well as the other NGA West 2 GMPEs - into the
        OQ-engine.  
\end{itemize}
