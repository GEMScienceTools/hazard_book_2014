%
% ..............................................................................
\section{Use of GMPEs in seismic hazard analysis}
% 
The standard process adopted for the implementation in the \gls{acr:oqe} 
of a \gls{groundshakingintensitymodel} requires a 
set of verification tables which contain values of ground-motion (or standard 
deviation) computed using a large number of combinations for the possible 
predictor variables. 

Table \ref{tab:verification} shows a simplified example of a \gls{acr:gsim} 
verification table.
% --------------------------------------------------------------------->>> Table
\begin{table}[!hb]
\centering
\begin{tabular}{|cccccc|}
\hline
\rowcolor{anti-flashwhite}
M & R & V$_{S,30}$ & IMT$_1$ & IMT$_2$ & ... \\
\hline 
val$_{1,1}$ & val$_{1,2}$ & val$_{1,3}$ & val$_{1,4}$ & val$_{1,5}$ & \\
val$_{2,1}$ & val$_{2,2}$ & val$_{2,3}$ & val$_{2,4}$ & val$_{2,5}$ & \\
... & & & & & \\
\hline
\end{tabular}
\caption{Schematic of a \gls{acr:gsim} verification table used in the 
\gls{acr:oqe}.}
\label{tab:verification}
\end{table}
% ---------------------------------------------------------------------<<< Table


Examples of verification tables are available on the OpenQuake-hazardlib Github
repository\footnote{
\href{https://github.com/gem/oq-hazardlib/tree/master/openquake/hazardlib/tests/gsim/data}{
https://github.com/gem/oq-hazardlib/tree/master/openquake/hazardlib/tests/gsim/data}
}.

Through these tables and an automated verification procedure, it is 
possible to achieve with a high confidence level the consistency between 
the original results and the corresponding values computed with the 
version of the \gls{acr:gsim} implemented.
%
% . . . . . . . . . . . . . . . . . . . . . . . . . . . . . . . . . . . . . . .
\subsection{Testing}
%
The progressively increasing complexity of 
\glspl{groundshakingintensitymodel}
is giving more and more emphasis and relevance to the validation between 
between the results provided by the \gls{acr:gsim} implemented in
\gls{acr:psha} codes and the results of original \gls{acr:gsim} 
implementations described in the scientific literature or directly 
provided by the authors.

GEM recommends contextually to the publication of \gls{acr:gsim} recommends
distributing as an electronic attachment the table of coefficients as well 
as of a set of verification tables (or a software which allows the generation 
of these tables). This can certainly improve the reproducibility of the 
new models proposed and most of all would improve the quality and 
robustness of the computed hazard.

In order to assist the user in this process of validating the implemented 
\gls{acr:gsim}, the \gls{acr:oqe} provides an automated procedure. 
%
% . . . . . . . . . . . . . . . . . . . . . . . . . . . . . . . . . . . . . . .
\subsection{Selection criteria}
The \gls{acr:oqe} does not provide tools supporting the \gls{acr:gsim}
selection process, which is now common practice in a PSHA model building 
process \parencite[see for example][]{delavaud2012}. Nevertheless, in order 
to ensure the higher level of consistency between the selection procedure and
the calculation of hazard it is recommendable to use in both the phases the 
same \gls{acr:gsim}. 

The modularity used to create the {acr:oqhl} promotes the use of 
\glspl{acr:gsim} into separate libraries. The %\citetitle{weatherill2014}
\parencite{weatherill2014} provides tools for the calculation of ground-motion
residuals and for the selection of \glspl{acr:gsim} models which build on top 
of the \gls{acr:oqe} hazard library. In this way the 
%
% ..............................................................................
\section{Calculation of ground motion fields}
%
The calculation of \glspl{groundmotionfield} is an essential step in the
calculation of hazard via an \gls{epsha} and for the calculation of losses 
in case of spatially distributed assets.
% . . . . . . . . . . . . . . . . . . . . . . . . . . . . . . . . . . . . . . .
\subsection{Spatial correlation of ground motion}
The spatial correlation of within-event residuals 
is a topic that is 
receiving more and more importance in the ground-motion community given 
its importance on the calculation of losses for portfolios of distributed 
assets \parencite{crowley2006}.
%
% ..............................................................................
\section{Modified GMPEs}
The \gls{acr:oqe} currently does not provide a straightforward methodology 
to modify currently implemented \glspl{acr:gsim}.
%
% . . . . . . . . . . . . . . . . . . . . . . . . . . . . . . . . . . . . . . .
\subsection{Adjusted equations}
%
% . . . . . . . . . . . . . . . . . . . . . . . . . . . . . . . . . . . . . . .
\subsection{Non-ergodic sigma}

