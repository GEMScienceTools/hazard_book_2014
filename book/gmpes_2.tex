%
% ..............................................................................
\section{Implementation and use of GMPEs in seismic hazard analysis}
The \gls{acr:oqe} contains a large set of \glspl{acr:gsim}
developed for different tectonic regions. 
%
Currently the engine includes \glspl{acr:gsim} for shallow earthquakes
in active tectonic regions, earthquakes in stable continental regions,
subduction regions and geothermal areas.

\glspl{acr:gsim} are implemented following a template model (in the 
Python jargon a base class) which defines the basic behaviour and 
describes the principles to be followed for their implementation.
%
Each \gls{acr:gsim} contains a definition of the independent parameters
used to describe the rupture, the site conditions, the rupture-site
distance metrics, the intensity measure types supported, the type of 
standard deviation provided, the tectonic region where the use of the 
\gls{acr:gsim} is recommended.

The main advantage of this approach is that \glspl{acr:gsim}, no matter 
which are their specific properties or features, behaves following 
a common standard. 
%
For example, this feature allowed the creation on top of the 
\gls{acr:gsim} library of a universal testing procedure, a standard 
applied to all the models implemented in the \gls{acr:oqe} which 
guarantees a baseline uniform quality assurance level. 

A second advantage of the developed library relates to its flexibility 
and modularity. Once the properties of main objects are defined, 
\glspl{acr:gsim} can be used interchangeably. 
%
The OpenQuake Ground Motion Toolkit \parencite{weatherill2014} for 
example builds on top of this library and, for example, includes 
tools for computing residuals given a dataset of recordings, for 
the calculation of trellis plots.
%
Figure \ref{fig:gsim_mag_scaling} shows the scaling of ground-motion 
versus magnitude for some of the \glspl{acr:gsim} implemented in the 
\gls{acr:oqe}.
%
The ground motion is computed for a site at a R$_{jb}$ distance of 
about 33 km with V$_{S,30}$ equal to 760 m/s from a rupture with a 
strike of dip of 45 degrees toward for two different values of rake 
(i.e. rupture mechanism).


% . . . . . . . . . . . . . . . . . . . . . . . . . . . . . . . . . . . > Figure
\begin{figure}[hb]
\centering
% left bottom right top
\includegraphics[trim = 23mm 0mm 23mm 5mm, clip, width=\textwidth]
    {./Pictures/gsim/mag_scaling_example.pdf}
\caption{Scaling of Peak Ground Acceleration as a function of magnitude 
    obtained by some of the \gls{acr:gsim} implemented in the \gls{acr:oqe}. 
    The ipython notebook used to generate this figure can be downloaded  
    \href{https://github.com/GEMScienceTools/hazard_book_2014/tree/lts/notebooks}{here}.}
\label{fig:gsim_mag_scaling}
\end{figure}
% . . . . . . . . . . . . . . . . . . . . . . . . . . . . . . . . . . . < Figure
%
% . . . . . . . . . . . . . . . . . . . . . . . . . . . . . . . . . . . . . . .
\subsection{Testing}
%
The progressively increasing complexity of 
\glspl{groundshakingintensitymodel}
is giving more and more emphasis and relevance to the validation between 
between the results provided by the \gls{acr:gsim} implemented in
\gls{acr:psha} codes and the results of original \gls{acr:gsim} 
implementations described in the scientific literature or directly 
provided by the authors.

The standard process adopted for the implementation in the 
\gls{acr:oqe} of a \gls{groundshakingintensitymodel} requires a 
set of verification tables each one containing values of 
ground-motion (or standard deviation) computed using a large 
number of combinations of the predictor variables. 

Table \ref{tab:verification} shows a simplified example of a 
\gls{acr:gsim} verification table; it consists of: a header 
line with (standard) names for each of the column, a number
of lines each one containing values of the predictor variables
plus the computed values of ground-motion intensity or standard
deviation.
% --------------------------------------------------------------------->>> Table
\begin{table}[!ht]
\centering
\begin{tabular}{|cccccc|}
\hline
\rowcolor{anti-flashwhite}
M & R & V$_{S,30}$ & IMT$_1$ & IMT$_2$ & ... \\
\hline 
val$_{1,1}$ & val$_{1,2}$ & val$_{1,3}$ & val$_{1,4}$ & val$_{1,5}$ & \\
val$_{2,1}$ & val$_{2,2}$ & val$_{2,3}$ & val$_{2,4}$ & val$_{2,5}$ & \\
... & & & & & \\
\hline
\end{tabular}
\caption{Schematic of a \gls{acr:gsim} verification table used in the 
\gls{acr:oqe}.}
\label{tab:verification}
\end{table}
% ---------------------------------------------------------------------<<< Table
Examples of verification tables are available on the OpenQuake-hazardlib Github
repository\footnote{
\href{https://github.com/gem/oq-hazardlib/tree/master/openquake/hazardlib/tests/gsim/data}{
https://github.com/gem/oq-hazardlib/tree/master/openquake/hazardlib/tests/gsim/data}
}.

Using these tables and an automated verification procedure implemented
in the \gls{acr:oqe}, it is possible to verify the consistency between 
the original results and the corresponding values computed with the 
version of the \gls{acr:gsim} implemented. 
%
On average we accept \gls{acr:oqe}

GEM recommends contextually to the publication of \gls{acr:gsim} recommends
distributing as an electronic attachment the table of coefficients as well 
as of a set of verification tables (or a software which allows the generation 
of these tables). This can certainly improve the reproducibility of the 
new models proposed and most of all would improve the quality and 
robustness of the computed hazard.
%
% . . . . . . . . . . . . . . . . . . . . . . . . . . . . . . . . . . . . . . .
\subsection{Selection criteria}
The \gls{acr:oqe} does not provide tools supporting the \gls{acr:gsim}
selection process, which is now common practice in a PSHA model building 
process \parencite[see for example][]{delavaud2012}. Nevertheless, in order 
to ensure the higher level of consistency between the selection procedure and
the calculation of hazard it is recommendable to use in both the phases the 
same \gls{acr:gsim}. 

The modularity used to create the {acr:oqhl} promotes the use of 
\glspl{acr:gsim} into separate libraries. The %\citetitle{weatherill2014}
\parencite{weatherill2014} provides tools for the calculation of ground-motion
residuals and for the selection of \glspl{acr:gsim} models which build on top 
of the \gls{acr:oqe} hazard library. In this way the 
%
% . . . . . . . . . . . . . . . . . . . . . . . . . . . . . . . . . . . . . . .
\subsection{Spatial correlation of ground motion}
The spatial correlation of within-event residuals is a topic that is 
receiving more and more importance in the ground-motion community given 
its importance on the calculation of losses for portfolios of distributed 
assets \parencite{crowley2006}.
