% OpenQuake Book Glossary 
% To cite a glossary element in a document:
%	\gls{seismicsourcedata}
%	\Gls{seismicsourcedata} - First initial is uppercase
%	\GLS{seismicsourcedata} - All initials are uppercase
%	\glspl{seismicsourcedata} - Plural
% To process the glossary:
% 	makeglossaries oqb

%
% ------- A
\newglossaryentry{areasource}{
	name = area source,
	description={PSHA source typology usually adopted to model distributed 
	seismicity. The rate of occurrence of seismicity is assumed uniform over
	the source area; this produces an hazard pattern consisting of a more or 
	less uniform patch resembling the shape of the polygon smoothed at the 
	borders}
}
%
% ------- B
\newglossaryentry{branch}{
	name = branch,
	plural= branches,
	description={
	The simplest element in a logic tree; it belongs to a 
	\gls{branchset} where it represents one possible option among a finite 
	number of alternatives. A branch is associated with a weight 
	value \citep{scherbaum2011} if the \gls{branchset} represents the 
	epistemic uncertainty on a parameter or a model when the \gls{branchset} 
	is used to specify alternative models (e.g. district \glspl{acr:fmd})
	}
}
\newglossaryentry{branchinglevel}{
	name = branching level,
	description={It indicates the position where a \gls{branchset} or a 
	\gls{branch} is located in a logic tree structure. For example, 
	in \gls{acr:oq} the first branching level of the 
	\gls{seismicsourcelogictree} always contains one or several 
	\glspl{initialseismicsourcemodel}
	}
}
\newglossaryentry{branchset}{
	name = branch set,
	description={The structure describing the epistemic uncertainty on 
	a specific parameter or model included in a logic tree structure. 
	It ensembles a number of \glspl{branch}, each one representing a 
	discrete alternative}
}
%
% ------- C
\newacronym{cpsha}{cPSHA}{Classical PSHA}
\newglossaryentry{configurationfile}{
	name =  configuration file,
	description = {
	Usually the file containing the information necessary to run a calculation
	in OpenQuake
	}
}
\newglossaryentry{complexfaultsource}{
	name = complex fault source,
	description={
	A source typology usually adopted to model subduction interface faults
	}
}
%
% ------- D
\newglossaryentry{seismichazarddisaggregation}{
	name =  seismic hazard disaggregation,
	description = {
	A methodology to investigate the contributions to a specific
	level of hazard in terms of fundamental variables commonly used
	to characterize seismic sources and ground motion models (e.g. 
	magnitude, source-site distance, \gls{epsilon}}
}
\newglossaryentry{disaggregationmatrix}{
	name =  disaggregation matrix,
	description = {
	A multi-dimensional matrix used to systematically store the contributions
	to a level of hazard to be disaggregated and that is specified by the 
	user.
	See also \gls{seismichazarddisaggregation}}
}
%
% ------- E
\newacronym{acr:erf}{ERF}{Earthquake\- Rup\-ture\- Forecast}
\newacronym{acr:epsha}{ePSHA}{Event-based PSHA}
%
\newglossaryentry{earthquakeruptureforecast}{
	name = earthquake rupture forecast,
	description={
	A list of all possible ruptures generated by all the sources included 
	in a seismic source model. Each element in the list contains: the rupture 
	geometry and the rupture probability of occurrence in a given time span. 
	%
	See also the definition available on the 
	\href{http://www.opensha.org/glossary-earthquakeRuptureForecast}
	{OpenSHA website}}
}
\newglossaryentry{earthquakeruptureforecastcalculator}{
	name = earthquake rupture forecast calculator,
	description={
	Calculator producing a \gls{seismicsourcemodel} from a 
	\gls{seismicsourcelogictree} 
	}
}
%
\newglossaryentry{epsilon}{
	name = epsilon,
	description={
	normalized residual of the ground motion}
}
%
\newglossaryentry{epsha}{
	name = event-based seismic hazard analysis,
	description={
        Calculation of seismic hazard through a Monte Carlo based procedure.
	}
}
%
% ------- F
%
% ------- G
\newacronym{acr:gem}{GEM}{Global Earthquake Model}
\newacronym{acr:gmpe}{GMPE}{Ground Motion Prediction Equation}
\newacronym{acr:gsim}{GSIM}{Ground Shaking Intensity Model}
\newacronym{acr:gmm}{GMM}{Ground Motion Model}

\newglossaryentry{gridsource}{
	name = grid source,
	description={
	PSHA source typology usually adopted to model distributed 
	seismicity. It's usually produced by a seismicity smoothing algorithm 
	(one of the most famous algorithm is the one proposed by 
	\citet{frankel1995})}
}
\newglossaryentry{groundmotionfield}{
	name = ground-motion field,
	description={An object describing the geographic distribution around 
	a rupture of a ground motion intensity measure}
}
\newglossaryentry{groundmotionfieldcalc}{
	name = ground-motion field calculator,
	description={An \gls{acr:oq} calculator that given a rupture computes the 
	geographic distribution of a ground motion intensity parameter. Currently
	OQ can generate ground motion fields using a \gls{acr:gmpe}}
}
\newglossaryentry{groundmotionlogictree}{
	name = ground-motion logic tree,
	description={Tool used to systematically describe the epistemic 
	uncertainties related to the ground motion models used in the 
	computation of hazard using a specific \gls{pshainputmodel}}
}
\newglossaryentry{groundshakingintensitymodel}{
    name=ground shaking intensity model,
    description={}i
    }
\newglossaryentry{groundmotionmodel}{
	name = ground-motion model,
	description={An object that given a rupture with specific properties
	computes the expected ground motion at the given site. In simplest case 
	a ground motion model corresponds to a \gls{groundmotionpredictioneq}. 
	In case of complex PSHA input models, the produced ground motion models 
	contains a set of \glspl{acr:gmpe}, one for each tectonic region considered.
	}
}
\newglossaryentry{groundmotionparameter}{
	name = ground-motion parameter,
	description={A scalar or vector quantity describing a relevant property
	of the shaking such as intensity (e.g. PGA or Spectral Acceleration) 
	or duration, equivalent number of cycles 
    \citep[see for example][]{hancock2005})
	}
}
\newglossaryentry{groundmotionpredictioneq}{
	name = ground-motion prediction equation,
	description={
		An equation that - given some fundamental parameters characterizing 
		the source, the propagation path and the site (in the simplest 
		case magnitude, distance and V$_\text{S,30}$) - computes the 
		value $GM$ of a (scalar) ground motion intensity parameter.
	}
}
\newglossaryentry{groundmotionsystem}{
	name = ground-motion system,
	description={An object containing a list of \gls{groundmotionlogictree}}
}
%
% ------- I 
\newacronym{acr:imt}{IMT}{Intensity Measure Type}
\newglossaryentry{initialseismicsourcemodel}{
	name = initial seismic source model,
	description={It's a \gls{seismicsourcemodel} included in the first 
	branching level of a seismic source logic tree}
}
\newglossaryentry{investigationtime}{
	name = investigation time,
	description={The time interval considered to calculate hazard; usually 
	it corresponds to 50 years}
}
%
% ------- L
\newglossaryentry{logictree}{
	name = logic tree,
	description={Data structure used to systematically describe uncertainties
	on parameters and models used in a PSHA study}
}
\newglossaryentry{logictreeprocessor}{
	name = logic tree processor,
	description={An OQ calculator that takes the PSHA Input Model and creates 
	many realisations of a \gls{seismicsourcemodel} and of a 
	\gls{groundmotionmodel}}
} 
\newacronym{acr:ltmcs}{LTMCS}{Logic Tree Monte Carlo Sampler}
%
% ------- M
\newacronym{acr:msr}{MSR}{Magnitude-Scaling Relationship}
\newacronym{acr:mfd}{MFD}{Magnitude-Frequency Distribution}
\newglossaryentry{magnitudefrequencydistribution}{
	name =  magnitude-frequency distribution,
	description = {
	It describes the density of earthquakes with a specific 
	magnitude occour. It can be continuonus or discrete. 
    One frequency-magnitude distribution frequently adopted in 
    \gls{acr:psha} is the double truncated Gutenberg-Richter distribution.
	}
}
%
% ------- O
\newglossaryentry{opensha}{
	name = OpenSHA,
	description = {OpenSHA is an open-source, advanced Java-based 
	platform for conducting Seismic Hazard Analysis - 
	(see \href{http://opensha.org}{OpenSHA website}). \gls{acr:oq}-hazard 
	relies on a distilled version of OpenSHA}
}
\newacronym{acr:oq}{OQ}{OpenQuake}
\newacronym{acr:oqe}{OQ-engine}{OpenQuake-engine}
\newacronym{acr:oqhl}{OQ-hazardlib}{OpenQuake-hazardlib}
%
% ------- N
\newacronym{acr:nrml}{NRML}{Natural hazard Risk Markup Language}
%
% ------- P
\newacronym{acr:pga}{PGA}{Peak Ground Acceleration}
\newacronym{acr:pgv}{PGV}{Peak Ground Velocity}
\newacronym{acr:psha}{PSHA}{Probabilistic Seismic Hazard Analysis}
\newglossaryentry{pshainputmodel}{
	name=PSHA input model, 
	description={Object containing the information necessary to describe 
	the seismic source and the ground motion models - plus the related 
	epistemic uncertainties}	
}
\newglossaryentry{psha}{
	name = probabilistic seismic hazard analysis, 
	description={A methodology to compute seismic hazard which takes into 
	account the contributions coming from all the sources of engineering 
    importance for a specified site}	
}
%
% ------- N
\newacronym{acr:qa}{QA}{Quality assurance}
%
% ------- R
\newglossaryentry{rupture}{
	name=earthquake rupture, 
	description={A 3D surface representing the 
	%
	See also the definition available on the 
	\href{http://www.opensha.org/glossary-earthquakeRupture}
	{OpenSHA website}
	}	
}
%
% ------- S
\newacronym{acr:sha}{SHA}{Seismic Hazard Analysis}
\newglossaryentry{seismicityhistory}{
	name = seismicity history,
	plural= seismicity histories,
	description = {An object containing a set ruptures  
	representative of the possible seismicity generated by the 
	sources in a \gls{seismicsourcemodel} during the investigation 
	time $t$
	}
}
\newglossaryentry{seismicityrate}{
	name = seismicity rate,
	description = {Number of events per unit of time (if not better 
	specified, the definition of a seismicity rate generally presumes 
	a time independent 
	}
}
\newglossaryentry{seismicsourcedata}{
	name = seismic source data,
	description={An object containing the information necessary 
	to completely describe a \gls{acr:psha} seismic source i.e. seismic 
	source type, position, geometry and seismicity occurrence 
	model}
}
\newglossaryentry{seismicsourcelogictree}{
	name = seismic source logic tree,
	description={Logic tree structure defined to describe in 
	structured and systematic way the epistemic uncertainties 
	characterizing the seismic source model. The first 
	branching level in the logic tree by definition contains one or
	several alternative \gls{initialseismicsourcemodel}}
}
\newacronym{acr:ssm}{SSM}{Seismic Source Model}
\newglossaryentry{seismicsourcemodel}{
	name = seismic source model,
	description={An object containing a list of \gls{seismicsourcedata}}
}
\newacronym{acr:scec}{SCEC}{Southern California Earthquake Center}
\newglossaryentry{seismicsourcesystem}{
	name = seismic source system,
	description={An object containing a list of \glspl{initialseismicsourcemodel}
	and the \gls{seismicsourcelogictree}}
}
\newglossaryentry{simplefaultsource}{
	name = simple fault source,
	description={
	A source typology usually adopted to model shallow structures with an
	uncomplicated geometry
	}
}
\newacronym{acr:ses}{SES}{Stochastic Event Set}
\newglossaryentry{stochasticeventset}{
	name = stochastic event set,
	description={An object containing one or many \glspl{seismicityhistory} 
	}
}
\newacronym{acr:sa}{S$_a$}{Spectral Acceleration}
%
% ------- T
\newglossaryentry{tectonicregion}{
	name = tectonic region,
	description = {A area on the topographic surface that can be considered 
	homogeneous in terms of tectonic properties such as the prevalent 
	seismogenic properties and/or the seismic wave propagation properties
	}
}
\newglossaryentry{temporaloccurrencemodel}{
	name = temporal occurrence model,
	description = {Usually a probabilistic model giving the probability of
	occurrence of an event in a specified \gls{investigationtime}
	}
}
%
% ------- U
\newacronym{acr:usgs}{USGS}{United States Geological Survey}
%
% ------- V 
\newglossaryentry{acr:vs30}{
	name = V$_{S,30}$,
	description = {Average shear wave velocity of the 
	materials in the uppermost 30m of the soil column}
}
