This chapter describes the mathematical framework for PSHA implemented by the OpenQuake-engine.
Two main assumptions are at the base of all PSHA calculators:
\begin{itemize}
	\item seismicity in a region is described by a collection of \textit{independent seismic sources}
	(i.e. the occurrence of an earthquake rupture in a source does not affect the probability of
	earthquake occurrence in the other sources)
	\item each source generates \textit{independent earthquake ruptures} (i.e. the occurrence of an
	earthquake rupture in a source does not affect the probability of occurrence of the other
	potential earthquake ruptures in the same source)
\end{itemize}

\section{Basic concepts}
The Classical, Event-Based, and Disaggregation analysis requires the definition
of two main components: the \textit{seismic source model}, that is a collection
of seismic sources describing the seismic activity in a region of interest, and
the \textit{ground motion model}, that is a mathematical relationship defining
the probability distribution of a ground motion parameter at a site given the
occurrence of an earthquake rupture.

The design of a seismic source model involves the specification of a number of
sources whose main parameters are the geometry, constraining the earthquake
rupture locations, and the \textit{magnitude-frequency distribution}, defining
the average annual occurrence rate over a magnitude range. A seismic source
model ($SSM$) can be therefore defined as a set of $I$ seismic sources
($Src$):
\begin{equation}
SSM = \left\{Src_{1}, Src_{2}, ..., Src_{I}\right\}
\end{equation}
Chapter \ref{chap:ssm} provides a detailed description of the different source
typologies supported by the OQ-engine. However, independently of the typology,
in a PSHA each source undergoes a discretization process which effectively
generates a number of distinct earthquake ruptures. A generic $i$-th source
defines therefore a set of $J$ earthquake ruptures:
\begin{equation}
Src_{i} = \left\{Rup_{i1}, Rup_{i2}, ..., Rup_{iJ}\right\}
\end{equation}


\section{Classical PSHA}
The classical PSHA analysis allows calculating the probabilities of exceeding,
at least once in a given time span, and at a given site, a set of ground motion
parameter levels considering all possible earthquake ruptures defined in a
seismic source model. Such a list of probability values is usually referred to
as \textit{hazard curve}.

We indicate with $P(X \ge x | T)$ the probability that a ground-motion parameter
$X$ exceeds, at least once in a time span $T$, a level $x$. $P(X \ge x | T)$ can
be computed as 1 minus the probability that none of the sources is causing a
ground motion exceedance. By assuming \textit{independent sources}, the
probability that none of the sources is causing an exceedance is equal to the
product of the probabilities that each source does not cause an exceedance, that
is:
\begin{align}
\label{eq:hazard_eq}
P(X \ge x | T) & =  1 - P_{src1}(X < x | T) * P_{src2}(X < x | T) * ... * P_{srcI}(X < x | T) \nonumber \\
		      & =  1 - \prod_{i=1}^{I} P_{src_{i}}(X < x | T)
\end{align}
where $P_{src_{i}}(X < x | T)$ is the probability that the $i$-th source is not
causing an exceedance and $I$ is the total number of sources in the source
model.

By further assuming each source generates \textit{independent earthquake
ruptures}, we can compute $P_{src_{i}}(X < x | T)$ as the product of the
probabilities that each rupture does not cause an exceedance, that is:
\begin{align}
\label{eq:prup_noexceed_src}
P_{src_{i}}(X < x | T) & = P_{rup_{i1}}(X < x | T) * P_{rup_{i2}}(X < x | T) * ... * P_{rup_{iJ}}(X < x | T) \nonumber \\
			        & = \prod_{j=1}^{J_{i}} P_{rup_{ij}}(X < x | T)
\end{align}
where $P_{rup_{ij}}(X < x | T)$ is the probability that the $j$-th rupture in the $i$-th source is not causing an exceedance and $J_{i}$ is
the total number of ruptures generated by the $i$-th source.

Intuitively, the fact that a rupture does not cause any
exceedance in a given time span T can be due to the fact that the rupture does not occur at all or that the
rupture occurs once but without causing an exceedance, or that the rupture occurs twice but both times
without causing an exceedance, and so on. Given that all these events are mutually exclusive, by using the
total probability theorem we can write:
\begin{align}
\label{eq:prup_noexceed_rup}
P_{rup_{ij}}(X < x | T) & = P_{rup_{ij}}(n = 0 | T) + P_{rup_{ij}}(n = 1 | T) * P(X < x | rup_{ij}) + \nonumber \\
                                &\quad	P_{rup_{ij}}(n = 2 | T) * P(X < x | rup_{ij})^{2}  + ... \nonumber \\
				 & = \sum_{k=0}^{\infty} P_{rup_{ij}}(k | T) * P(X < x | rup_{ij}) ^ {k} 
\end{align}
where $P_{rup_{ij}}(k | T)$ is the probability that the $j$-th rupture in the $i$-th source is occurring $k$ times in time span $T$ and
$P(X < x | rup_{ij})$ is the conditional probability that parameter $X$ is not exceeding level $x$ given an
occurrence of $rup_{ij}$. 

Relying therefore on the assumptions of independent sources and independent earthquake ruptures generated by each source, we can compute
the probability of at least one ground motion exceedance as:
\begin{align}
\label{eq:hazard_eq_ind_srcs_rups}
P(X \ge x | T) & =  1 - \prod_{i=1}^{I} \prod_{j=1}^{J_{i}} P_{rup_{ij}}(X < x | T) \nonumber \\
                     & = 1 - \prod_{i=1}^{I} \prod_{j=1}^{J_{i}} \sum_{k=0}^{\infty} P_{rup_{ij}}(k | T) * P(X < x | rup_{ij}) ^ {k} 
\end{align}
It is worth noticing how equation \ref{eq:hazard_eq_ind_srcs_rups} allows the calculation of hazard curves for any
temporal occurrence model. More precisely, only the probability of the number of rupture occurrences $P_{rup_{ij}}(k | T)$
depends on the temporal occurrence model, while the procedure for calculating the final probability of ground motion exceedance
remains the same. This also means that is possible to define in the same source model sources with different temporal occurrence
models (for instance Poissonian and Brownian Passage Time distributions).

\subsection{Poissonian source model}
If we now assume a source model to consist only of Poissonian sources, we can write $P_{rup_{ij}}(k | T)$ for every rupture as:
\begin{equation}
\label{eq:poisson_pd}
P_{rup_{ij}}(k | T) = e^{-\nu_{ij} T} \frac{(\nu_{ij} T) ^ {k}}{k!}
\end{equation}
where $\nu_{ij}$ is the average annual occurrence rate for the $j$-th rupture in the $i$-th source. We can then place equation \ref{eq:poisson_pd} in \ref{eq:prup_noexceed_rup} and thus write:
\begin{align}
\label{eq:prup_noexceed_rup_pois_v0}
P_{rup_{ij}}(X < x | T) & = \sum_{k=0}^{\infty} e^{-\nu_{ij} T} \frac{(\nu_{ij} T) ^ {k}}{k!} * P(X < x | rup_{ij}) ^ {k} \nonumber \\
				 & =  e^{-\nu_{ij} T} \sum_{k=0}^{\infty} \frac{(\nu_{ij} T * P(X < x | rup_{ij})) ^ {k}}{k!}
\end{align}
Making use of the property:
\begin{equation}
e^{x} = \sum_{k=0} ^ {\infty} \frac{x^{k}}{k!}
\end{equation}
we can rewrite \ref{eq:prup_noexceed_rup_pois_v0} as:
\begin{align}
\label{eq:prup_noexceed_rup_pois_v1}
P_{rup_{ij}}(X < x | T) & = e^{-\nu_{ij} T} e ^ {\nu_{ij} T * P(X < x | rup_{ij})} \nonumber \\
				 & = e^{-\nu_{ij} T * (1 - P(X < x | rup_{ij}))} \nonumber \\
				 & = e^{-\nu_{ij} T * P(X \ge x | rup_{ij})}
\end{align}
By now recognizing that, according to the Poissonian distribution, the
probability of at least one occurrence (that is one or more) in a time span $T$
of $rup_{ij}$ is:
\begin{equation}
P_{rup_{ij}}(n \ge 1 | T) = 1 -  e^{-\nu_{ij} T}
\end{equation}
we can write equation \ref{eq:prup_noexceed_rup_pois_v1} as:
\begin{equation}
\label{eq:prup_noexceed_rup_pois_v2}
P_{rup_{ij}}(X < x | T) = (1 - P_{rup_{ij}}(n \ge 1 | T))^{P(X \ge x | rup_{ij})}
\end{equation}
By placing equation \ref{eq:prup_noexceed_rup_pois_v2} in \ref{eq:hazard_eq_ind_srcs_rups}, we can
write:
\begin{equation}
\label{eq:hazard_eq_poiss}
P(X \ge x | T) =  1 - \prod_{i=1}^{I} \prod_{j=1}^{J_{i}} (1 - P_{rup_{ij}}(n \ge 1 | T))^{P(X \ge x | rup_{ij})}
\end{equation}
Equation \ref{eq:hazard_eq_poiss} is used by the OQ-engine for the calculation of hazard curves when performing
Classical PSHA with a Poissonian source model. To our knowledge, this equation has been first proposed by \citet{field2003},
derived from the traditional rate-based formulation converted in terms of probabilities (their equation A8).
Instead, we derive it from the assumptions of a source model consisting of independent sources, independent
earthquake ruptures generated by each source, and ruptures obeying to a Poissonian temporal occurrence model.

\subsection{Equivalence with the rate-based equation}
It is worth noticing how equation \ref{eq:hazard_eq_poiss} is equivalent to the more traditional rate-based
hazard equation (\cite{mcguire1995}). Indeed, by assuming ground motion occurrence to follow a Poissonian
distribution in time, and indicating with $\nu$ the mean annual rate of exceeding a ground motion level x,
we can write:
\begin{equation}
\label{eq:hazard_eq_rate}
P(X \ge x | T) = 1 - e ^ {- \nu T}
\end{equation}
We can also rewrite \ref{eq:hazard_eq_poiss} as:
\begin{align}
\label{eq:hazard_eq_prob}
P(X \ge x | T) &= 1 - \prod_{i=1}^{I} \prod_{j=1}^{J_{i}} (1 - P_{rup_{ij}}(n \ge 1 | T))^{P(X \ge x | rup_{ij})} \nonumber \\
		      &= 1 - \prod_{i=1}^{I} \prod_{j=1}^{J_{i}} e^{-\nu_{ij} T * P(X \ge x | rup_{ij})} \nonumber \\
		      & = 1 - e ^ {- \sum_{i=1}^{I} \sum_{j=1}^{J_{i}} \nu_{ij} T * P(X \ge x | rup_{ij})}
\end{align}
The equivalence between equations \ref{eq:hazard_eq_rate} and \ref{eq:hazard_eq_prob} is possible if and only if:
\begin{align}
\label{eq:equivalence_condition_0}
\nu  =  \sum_{i=1}^{I} \sum_{j=1}^{J_{i}} \nu_{ij} * P(X \ge x | rup_{ij})
\end{align}

Assuming now, for the sake of simplicity, that a rupture is completely characterized by magnitude and distance from a site, we
can write the rate of occurrence of the $j$-th rupture as:
\begin{equation}
\label{eq:rup_rate}
\nu_{ij} = \nu_{i} * f_{i}(m, r)
\end{equation}
where $\nu_{i}$ is the total occurrence rate for the $i$-th source, and $f_{i}(m, r)$ is the probability, for the $i$-th source, of
generating a rupture of magnitude $m$ and distance $r$. By placing \ref{eq:rup_rate} in \ref{eq:equivalence_condition_0} and
by replacing the discrete summation over ruptures with a continuous integral over magnitude and distance, we can write:
\begin{align}
\nu = \sum_{i=1}^{I} \nu_{i} \iint f_{i}(m, r) P(X \ge x | m, r)\,dm \,dr
\end{align}
which is the traditional equation for calculating ground motion exceedance rates
\citep{mcguire1995}.

\section{Event-based PSHA}
<<<<<<< HEAD
The goal of an Event-based PSHA is to simulate seismicity in a region as described by a source model
and to simulate ground shaking on a set of locations accordingly with a ground motion model. In both
cases, simulation involves a Monte Carlo (i.e. random) sampling procedure.

Seismicity is simulated by generating a \textit{stochastic event set} (also known as \textit{synthetic catalog})
for a given time span $T$. For each rupture generated by a source,
the number of occurrences in a time span $T$ is simulated by sampling the corresponding probability
distribution as given by $P_{rup}(k | T)$. A stochastic event set is therefore a \textit{sample}
of the full population of ruptures as defined by a seismic source model. Each rupture is present zero, one or
more times, depending on its probability. Symbolically, we can define a stochastic event set ($SES$) as:
\begin{align}
SES(T) = \left\{k \times rup,\;k\sim P_{rup}(k | T)\;\;\forall\;rup\;in\;Src\;\forall\;Src\;in\;SSM\right\}
\end{align}
where $k$, the number of occurrences, is a random sample of $P_{rup}(k | T)$, and $k \times rup$ means
that rupture $rup$ is repeated $k$ times in the stochastic event set.

Given an earthquake rupture, the simulation of ground shaking values on a set of locations $\bm{x}=(x_{1}, x_{2}, ..., x_{N})$
forms a \textit{ground motion field}. In a Event-based PSHA, for each rupture in a stochastic event set,
the ground motion field is obtained by sampling the probability distribution defined by the ground motion model.
As described in Chapter \ref{chap:gmpes}, the ground motion distribution at a site is assumed to be a Normal
distribution. The aleatory variability is described in terms of an \textit{inter-event} (also known as \textit{between-events})
standard deviation ($\tau$) and \text{intra-event} (also known as \textit{within-event}) standard deviation ($\sigma$).
The simulation of a ground motion field is therefore the result of the summation of three terms, the logarithmic mean of the
ground motion distribution:
\begin{equation}
\bm\mu = (\mu_{1}, \mu_{2}, ..., \mu_{N})
\end{equation}
the inter-event variability:
\begin{equation}
\bm\eta = (\eta, \eta, ..., \eta),\;where\;\eta\sim N(0, \tau)
\end{equation}
and the intra-event variability:
\begin{align}
\bm\epsilon = (\epsilon_{1}, \epsilon_{2}, ..., \epsilon_{N}) \sim N(\bm{0}, \bm\Sigma)
\end{align}
where:
\begin{align}
\bm\Sigma = 
\begin{bmatrix}
\sigma_{1}^2&\sigma_{1}\sigma_{2}\rho_{12}&\cdots &\sigma_{1}\sigma_{N}\rho_{1N} \\
\sigma_{2}\sigma_{1}\rho_{21}&\sigma_{2}^2&\cdots &\sigma_{2}\sigma_{N}\rho_{2N} \\
\vdots & \vdots & \ddots & \vdots\\
\sigma_{N}\sigma_{1}\rho_{N1}&\sigma_{N}\sigma_{2}\rho_{N2}&\cdots &\sigma_{N}^2
\end{bmatrix}
\end{align}
%
It is worth noticing how the inter-event variability, uniform for all sites
given an earthquake rupture, is drawn from an univariate normal distribution of
mean 0 and standard deviation $\tau$, while the intra-event variability is a
random sample of a multivariate normal distribution of mean $\bm{0}$ and
covariance matrix $\bm\Sigma$.  $\bm\Sigma$ is a diagonal matrix when
considering no correlation in the intra-event variability, while it has non-zero
off-diagonal elements when considering a correlation model ($\rho$) for the
intra-event aleatory variability.

%
%
\subsection{Calculation of hazard curves from ground motion fields}
%
The ground motion fields simulated for each rupture in a stochastic event set
can be used to compute hazard curves. Indeed, indicating with $T_{0}$ the
duration associated with a stochastic event set, and with $K$ the number of
ground motion fields (and associated ruptures) simulated in time $T_{0}$, we can
compute the rate of exceedance of a ground motion level $x$ at a site as
(\cite{ebel1999}):
\begin{align}
\label{eq:rate_from_gmfs}
\nu = \frac{\sum_{k=1}^{K}H(x_{k} - x)}{T_{0}}
\end{align}
where $H$ is the Heaviside function and $x_{k}$ is the ground motion parameter value at the considered site associated
with the $k$-th ground motion field. The exceedance rate obtained from equation \ref{eq:rate_from_gmfs} can then be
used to compute the probability of at least one occurrence in any time span $T$, accordingly with the Poissonian distribution,
using equation \ref{eq:hazard_eq_rate}. This approach is possible when the source model from which the stochastic
event set is generated is Poissonian.

As the stochastic event set duration $T_{0}$ increases, equation \ref{eq:rate_from_gmfs}
provides an increasingly more accurate estimate of the actual rate of exceedance. A larger $T_{0}$ can be achieved
not only by simulating a single stochastic event set with longer duration, but also by simulating multiple
stochastic event sets. These can then be joined together to form a stochastic event which has a large enough
duration to provide a stable estimate of the rates of ground motion exceedance.

\section{Disaggregation}
The disaggregation analysis allows investigating how the different earthquake ruptures defined in a source model
contribute to the probability of exceeding a certain ground motion level $x$ at a given site. Given the very
large number of earthquake ruptures associated with a source model, contributions cannot be investigated on a rupture by
rupture basis but a classification scheme is used instead. Ruptures are classified in terms of the following parameters:
\begin{itemize}
	\item magnitude ($M$)
	\item distance to rupture surface-projection (Joyner-Boore distance) ($r_{jb}$)
	\item longitude and latitude of rupture surface-projection closest point ($\lambda$, $\phi$)
	\item tectonic region type ($TRT$)
\end{itemize}
For each earthquake rupture, the associated conditional probability of ground motion exceedance -- $P(X \ge x | T, rup)$ -- is computed
for different $\epsilon$ bins, where $\epsilon$ is the difference, in terms of number of total standar deviations,
between $x$ and the mean ground motion $\mu$ as predicted by the ground motion model, that is:
\begin{equation}
\epsilon = \frac{x - \mu}{\sigma_{total}}
\end{equation}
The disaggregation in terms of $\epsilon$ allows investigating how the different regions of the ground motion distributions
contribute to the probability of exceedance.

The rupture parameters ($M$, $r_{jb}$, $\lambda$, $\phi$, $TRT$) together with the $\epsilon$ parameter effectively
create a 6-dimensional model space which, discretized into a number of bins, is used to classify the probability
of exceedance for different combination of rupture parameters and $\epsilon$ values.

For a given model space bin $\bm{m} = (M, r_{jb}, \lambda, \phi, TRT, \epsilon)$ the probability of
exceeding level $x$ at least once in a time span $T$ is computed using equation \ref{eq:hazard_eq_ind_srcs_rups}, that is:
\begin{align}
\label{eq:disagg}
P(X > x | T, \bm{m}) =
	1 - \prod_{i=1}^{I}\prod_{j=1}^{J_{i}}
	\begin{cases} P_{rup_{ij}}(X < x | T) & \mbox{if}\;rup_{ij}\;\in\;\bm{m}\\
			      1 & \mbox{if}\;rup_{ij}\;\notin\;\bm{m}
	\end{cases}
\end{align}
In other words, if a rupture belongs to the considered bin, then the probability of not causing a ground motion exceedance is computed
according to equation \ref{eq:hazard_eq_ind_srcs_rups}, otherwise the probability is 1 (that is, given that the rupture does not belong to the bin it can never cause a ground motion exceedance).

\subsection{Disaggregation histograms}
Disaggregation values as given by equation \ref{eq:disagg} can be aggregated in order to investigate earthquake rupture
contributions over a reduced model space. The following disaggregation histograms are provided by the OpenQuake-engine.\\
Magnitude disaggregation:
\begin{equation}
P(X > x | T, M) = 1 -\prod_{r_{jb}}\prod_{\lambda}\prod_{\phi}\prod_{TRT} \prod_{\epsilon}(1 - P(X > x | T, \bm{m}))
\end{equation}
Distance disaggregation:
\begin{equation}
P(X > x | T, r_{jb}) = 1 -\prod_{M}\prod_{\lambda}\prod_{\phi}\prod_{TRT} \prod_{\epsilon}(1 - P(X > x | T, \bm{m}))
\end{equation}
Tectonic region type disaggregation:
\begin{equation}
P(X > x | T, TRT) = 1 -\prod_{M}\prod_{r_{jb}}\prod_{\lambda}\prod_{\phi} \prod_{\epsilon}(1 - P(X > x | T, \bm{m}))
\end{equation}
Magnitude-Distance disaggregation:
\begin{equation}
P(X > x | T, M, r_{jb}) = 1 -\prod_{\lambda}\prod_{\phi}\prod_{TRT}\prod_{\epsilon}(1 - P(X > x | T, \bm{m}))
\end{equation}
Magnitude-Distance-Epsilon disaggregation:
\begin{equation}
P(X > x | T, M, r_{jb}, \epsilon) = 1 -\prod_{\lambda}\prod_{\phi}\prod_{TRT}(1 - P(X > x | T, \bm{m}))
\end{equation}
Longitude-Latitude disaggregation:
\begin{equation}
P(X > x | T, \lambda, \phi) = 1 -\prod_{M}\prod_{r_{jb}}\prod_{TRT}\prod_{\epsilon}(1 - P(X > x | T, \bm{m}))
\end{equation}
Longitude-Latitude-Magnitude disaggregation:
\begin{equation}
P(X > x | T, \lambda, \phi, M) = 1 -\prod_{r_{jb}}\prod_{TRT}\prod_{\epsilon}(1 - P(X > x | T, \bm{m}))
\end{equation}
Longitude-Latitude-Tectonic Region Type disaggregation:
\begin{equation}
P(X > x | T, \lambda, \phi, TRT) = 1 -\prod_{M}\prod_{r_{jb}}\prod_{\epsilon}(1 - P(X > x | T, \bm{m}))
\end{equation}
All the above equations are based on the assumption that earthquake ruptures in different bins are independent, therefore
probabilities can be aggregated by using the multiplication rule for independent events. The probability of a ground motion
exceedance over a reduced model space is computed simply as 1 minus the probabilty of non-exceedance over the remaining
model space dimensions.

\subsection{Comparison between OpenQuake-engine disaggregation and \textit{traditional} disaggregation}
The traditional disaggregation analysis as commonly known in literature (e.g. \cite{bazzurro1999}) differs from the one
provided by the OpenQuake-engine. Indeed, a disaggregation analysis tipically provides the conditional probability of
observing an earthquake scenario of given properties (magnitude, distance, epsilon, ...) given that a ground motion
exceedance is occurred, which can be written (following the same notation used in this chapter) as:
\begin{align}
\label{eq:disagg_class}
P(\bm{m} | X > x)
\end{align}
On the contrary, the OpenQuake-engine (as described in equation \ref{eq:disagg}) provides the conditional probability of observing
at least one ground motion exceedance in a time span $T$ given the occurrence of earthquake ruptures of given properties $\bm{m}$,
that is:
\begin{align}
\label{eq:disagg_oq}
P(X > x | T, \bm{m})
\end{align}
The probabilities given in equations \ref{eq:disagg_class} and \ref{eq:disagg_oq} are clearly different. Indeed, for different $\bm{m}$, values given
by equation \ref{eq:disagg_class}  must sum up to 1, while this is not the case for equation \ref{eq:disagg_oq}. For the former equation different $\bm{m}$ represent mutually exclusive events, while for the latter they represent independent events.

When considering a Poissonian source model it is possible however to derive equation \ref{eq:disagg_class} from equation \ref{eq:disagg_oq}. Indeed,
indicating with $\nu_{\bm{m}}$ the rate of ground motion exceedance ($X>x$) associated with earthquake ruptures of properties $\bm{m}$ and with $\nu$ the rate of ground motion exceedance associated with all earthquake ruptures, we can write equation \ref{eq:disagg_class} as:
\begin{equation}
P(\bm{m} | X > x) = \frac{\nu_{\bm{m}}}{\nu}
\end{equation}
By solving the Poissonian equation \ref{eq:hazard_eq_rate} for the rate of exceedance, we can write $\nu_{\bm{m}}$ as:
\begin{equation}
\nu_{\bm{m}} = - \frac{\ln(1 - P(X > x | T, \bm{m}))}{T}
\end{equation}
$\nu$ can be obtained using the same equation above but considering $P(X > x | T)$ instead of $P(X > x | T, \bm{m})$, where $P(X > x | T)$ is
obtained by aggregating, using the multiplicative rule, the probabilities over the different $\bm{m}$, that is:
\begin{equation}
P(X > x | T) = 1 - \prod_{\bm{m}}(1 - P(X > x | T, \bm{m}))
\end{equation}
By computing $\nu_{\bm{m}}$ and $\nu$ from $P(X > x | T, \bm{m})$ it is hence possible to obtain the more traditional disaggregation results as given in equation \ref{eq:disagg_class}.