This chapter contains a general introduction to \gls{acr:sha} and a 
description of the main concepts needed to understand the hazard calculation 
methodologies implemented in the OpenQuake-engine.

%\Gls{psha} is a methodology to compute the probability that ground motion, 
%for a site or region, will exceed a given level of intensity in a specified 
%time period. Originally formulated in the classic works of \cite{cornell1968} 
%and \cite{esteva1968}, the fundamental theory of \gls{acr:psha} has remained 
%robust. 
%Over the last four decades there have been many developments that have 
%increased 
%the accuracy and rigour of the process, particularly with respect to the 
%treatment of uncertainties.
%
% ..............................................................................
\section{Common concepts}
%
\subsection{Magnitude-frequency distributions}
\index{Magnitude-frequency distribution}
%
In statistics a distribution represents an arrangment of the 

The OpenQuake-engine supports the following \glspl{acr:mfd}:
\begin{itemize}
    \item Double-truncated Gutenberg-Richter;
    \item Evenly-spaced;
    \item Characteristic \cite{youngs1985}
\end{itemize}
%
% . . . . . . . . . . . . . . . . . . . . . . . . . . . . . . . . . . . . . . .
\subsection{Magnitude-scaling relationships}
%
\glspl{acr:msr} - a fundamental component of a seismic hazard analysis - are 
empirical functions linking independent an independent variable (e.g. 
magnitude) to a parameter describing some properties of the earthquake 
rupture geometry (e.g. area or length).
%
% . . . . . . . . . . . . . . . . . . . . . . . . . . . . . . . . . . . . . . .
\subsection{Ground-motion prediction equations}
\index{Ground-motion prediction equation}
A \gls{acr:gmpe} is an empirical relationship which associates a number of 
predictive 
%
% . . . . . . . . . . . . . . . . . . . . . . . . . . . . . . . . . . . . . . .
\subsection{Local soil conditions}
%
% ..............................................................................
\section{Methodologies for seismic hazard analysis}
\index{Seismic hazard analysis}
The hazard integral as implemented in the \gls{acr:oqe}

\cite{cornell1968}
\cite{mcguire2004}
%
% . . . . . . . . . . . . . . . . . . . . . . . . . . . . . . . . . . . . . . .
\subsection{Scenario-based Hazard Analysis}
\index{Seismic hazard analysis!Scenario-based}
%
% . . . . . . . . . . . . . . . . . . . . . . . . . . . . . . . . . . . . . . .
%
\subsection{Classical Probabilistic Seismic Hazard Analysis}
\index{Seismic hazard analysis!Classical PSHA}
%
% . . . . . . . . . . . . . . . . . . . . . . . . . . . . . . . . . . . . . . .
\subsection{Event-Based Probabilistic Seismic Hazard Analysis}
\index{Seismic hazard analysis!Event-based PSHA}
%
% . . . . . . . . . . . . . . . . . . . . . . . . . . . . . . . . . . . . . . .
\subsection{Disaggregation}
\index{Seismic hazard analysis!Disaggregation}
%
% . . . . . . . . . . . . . . . . . . . . . . . . . . . . . . . . . . . . . . .
\section{What's missing}
\subsection{Vector based PSHA}
\index{Seismic hazard analysis!Vector-based PSHA}
\subsection{PFDHA}
\index{Seismic hazard analysis!Probabilistic fault displacement hazard analysis}
\subsection{PSHA using site-specific transfer functions}
